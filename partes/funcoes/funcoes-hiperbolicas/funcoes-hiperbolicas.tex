\section{Funções Hiperbólicas}

\subsection{Definições das Funções Hiperbólicas}
\begin{frame}
  \begin{columns}[onlytextwidth]
    \begin{column}{0.49\textwidth}\vspace*{-0.5cm}
      \begin{definition}[Funções Hiperbólicas]
        As relações hiperbólicas são definidas na hipérbole unitária:\vspace*{-0.35cm}
        \begin{equation*}
          x^{2} - y^{2} = 1.\vspace*{-0.35cm}
        \end{equation*}
        Para um setor de área $\alpha/2$, definimos:
        \begin{enumerate}
          \item $\displaystyle\senh{\alpha} = \frac{e^{\alpha}-e^{-\alpha}}{2}$
          \item $\displaystyle\cosh{\alpha} = \frac{e^{\alpha}+e^{-\alpha}}{2}$
          \item $\displaystyle\tanh{\alpha} = \frac{\senh{\alpha}}{\cosh{\alpha}} = \frac{e^{\alpha}-e^{-\alpha}}{e^{\alpha}+e^{-\alpha}}$
          \item $\displaystyle\coth{\alpha} = \frac{\cosh{\alpha}}{\senh{\alpha}} = \frac{e^{\alpha}+e^{-\alpha}}{e^{\alpha}-e^{-\alpha}}$
          \item $\displaystyle\sech{\alpha} = \frac{1}{\cosh{\alpha}} = \frac{2}{e^{\alpha}+e^{-\alpha}}$
          \item $\displaystyle\csch{\alpha} = \frac{1}{\senh{\alpha}} = \frac{2}{e^{\alpha}-e^{-\alpha}}$
        \end{enumerate}
      \end{definition}
    \end{column}
    \begin{column}{0.49\textwidth}\vspace*{-0.5cm}
      \begin{figure}
        \includefigure[width=\textwidth]{hiperbole-unitaria.pdf}
      \end{figure}
    \end{column}
  \end{columns}
\end{frame}

\subsection{Função Seno Hiperbólico e Sua Inversa}
\begin{frame}
  \begin{columns}[onlytextwidth]
    \begin{column}{0.49\textwidth}\vspace{-0.5cm}
      \begin{definition}[Seno Hiperbólico]
        A função seno hiperbólico é definida como $f:\R\rightarrow\R$, que a cada $x\in\R$ associa o número real
        \begin{equation*}
          f(x) = \senh{x}
        \end{equation*}
      \end{definition}
      \begin{highlight}
        \textbf{Temos que a função seno possui:}
        \begin{itemize}
          \item $\Domain{f} = \R$;
          \item $\Image{f} = \R$;
          \item Função ímpar;
          \item<2> Arco seno hiperbólico: $f^{-1}:\R\rightarrow\R$\small
          \begin{equation*}
            f^{-1}(x)=\asenh{x} = \ln{\left(x + \sqrt{x^{2}+1}\right)}
          \end{equation*}
        \end{itemize}
      \end{highlight}
    \end{column}
    \begin{column}{0.49\textwidth}\vspace*{-0.5cm}
      \begin{figure}
        \includefigure<1>[width=\textwidth]{funcao-seno-hiperbolico.pdf}
        \includefigure<2>[width=\textwidth]{funcao-arco-seno-hiperbolico.pdf}
      \end{figure}
    \end{column}
  \end{columns}
\end{frame}

\subsection{Função Cosseno Hiperbólico e Sua Inversa}
\begin{frame}
  \begin{columns}[onlytextwidth]
    \begin{column}{0.49\textwidth}\vspace{-0.5cm}
      \begin{definition}[Cosseno Hiperbólico]
        A função cosseno hiperbólico é definida \\como $f:\R\rightarrow\R$, que a cada $x\in\R$ associa o número real
        \begin{equation*}
          f(x) = \cosh{x}
        \end{equation*}
      \end{definition}
      \begin{highlight}
        \textbf{Temos que a função seno possui:}
        \begin{itemize}
          \item $\Domain{f} = \R$, $\Image{f} = \R$;
          \item Função par;
          \item< 2- > Admite inversa apenas se restrita: $f:[0,+\infty)\rightarrow[1,+\infty)$;
          \item<3> Arco cosseno hiperbólico: $f^{-1}:[1,+\infty)\rightarrow[0,+\infty)$\small\vspace*{-0.2cm}
          \begin{equation*}
            f^{-1}(x)=\acosh{x} = \ln{\left(x + \sqrt{x^{2}-1}\right)}\vspace*{-0.2cm}
          \end{equation*}
        \end{itemize}
        \vspace*{-0.5cm}
      \end{highlight}
    \end{column}
    \begin{column}{0.49\textwidth}\vspace*{-0.5cm}
      \begin{figure}
        \includefigure<1>[width=\textwidth]{funcao-cosseno-hiperbolico.pdf}
        \includefigure<2>[width=\textwidth]{funcao-cosseno-hiperbolico-restrito.pdf}
        \includefigure<3>[width=\textwidth]{funcao-arco-cosseno-hiperbolico.pdf}
      \end{figure}
    \end{column}
  \end{columns}
\end{frame}

\subsection{Função Tangente Hiperbólico e Sua Inversa}
\begin{frame}
  \begin{columns}[onlytextwidth]
    \begin{column}{0.49\textwidth}\vspace{-0.5cm}
      \begin{definition}[Tangente Hiperbólica]
        A função tangente hiperbólica é definida como $f:\R\rightarrow\R$, que a cada $x\in\R$ associa o número real
        \begin{equation*}
          f(x) = \tanh{x}
        \end{equation*}
      \end{definition}
      \begin{highlight}
        \textbf{Temos que a função seno possui:}
        \begin{itemize}
          \item $\Domain{f} = \R$, $\Image{f} = (-1,1)$;
          \item Função ímpar;
          \item<2> Arco tangente hiperbólica: $f^{-1}:(-1,1)\rightarrow\R$\small
          \begin{equation*}
            f^{-1}(x)=\atanh{x} = \frac{1}{2}\ln{\left(\frac{1+x}{1-x}\right)}
          \end{equation*}
        \end{itemize}
      \end{highlight}
    \end{column}
    \begin{column}{0.49\textwidth}\vspace*{-0.5cm}
      \begin{figure}
        \includefigure<1>[width=\textwidth]{funcao-tangente-hiperbolica.pdf}
        \includefigure<2>[width=\textwidth]{funcao-arco-tangente-hiperbolica.pdf}
      \end{figure}
    \end{column}
  \end{columns}
\end{frame}
