\section{Simetrias, Funções Pares e Ímpares}
\begin{frame}
  \begin{definition}[Simetria em Relação ao Eixo $x$]
    Uma curva plana é simétrica em relação ao eixo $x$ se, para cada ponto $(x,\,y)$ do seu gráfico, o ponto $(x,\,-y)$ também está no gráfico.
  \end{definition}
  \begin{figure}
    \includefigure[width=0.9\textwidth]{simetria-x.pdf}
  \end{figure}
\end{frame}

\begin{frame}
  \begin{definition}[Simetria em Relação ao Eixo $y$]
    Uma curva plana é simétrica em relação ao eixo $y$ se, para cada ponto $(x,\,y)$ do seu gráfico, o ponto $(-x,\,y)$ também está no gráfico.
  \end{definition}
  \begin{figure}
    \includefigure[width=0.95\textwidth]{simetria-y.pdf}
  \end{figure}
\end{frame}

\begin{frame}
  \begin{definition}[Simetria em Relação à Origem]
    Uma curva plana é simétrica em relação à origem se, para cada ponto $(x,\,y)$ do seu gráfico, o ponto $(-x,\,-y)$ também está no gráfico.
  \end{definition}
  \begin{figure}
    \includefigure[width=0.95\textwidth]{simetria-origem.pdf}
  \end{figure}
\end{frame}

\begin{frame}
  \begin{theorem}[Testes de Simetria]
    Uma curva é simétrica em relação à origem se, para cada ponto $(x,\,y)$ do seu gráfico, o ponto $(-x,\,-y)$ também está no gráfico.
    \begin{enumerate}
      \item Uma curva plana é simétrica em relação ao eixo $y$ se, e somente se, substituindo-se $x$ por $-x$ em sua equação, obtém-se uma equação equivalente;
      \item Uma curva plana é simétrica em relação ao eixo $x$ se, e somente se, substituindo-se $y$ por $-y$ em sua equação, obtém-se uma equação equivalente;
      \item Uma curva plana é simétrica em relação à origem $y$ se, e somente se, substituindo-se $x$ por $-x$ e substituindo-se $y$ por $-y$ em sua equação, obtém-se uma equação equivalente.
    \end{enumerate}
  \end{theorem}
  \begin{columns}[onlytextwidth]
    \begin{column}{0.49\textwidth}
      \vspace*{-0.15cm}
      \begin{example-highlight}
        Classifique as curvas definidas pelas funções a seguir com respeito à sua simetria.
        \begin{columns}[onlytextwidth]
          \begin{column}{0.49\textwidth}\vspace*{-0.35cm}
            \begin{enumerate}
              \item $f(x)=\dfrac{1}{x^{3}}$
            \end{enumerate}
          \end{column}
          \begin{column}{0.49\textwidth}\vspace*{-0.35cm}
            \begin{enumerate}\setcounter{enumi}{1}
              \item $g(x)=x^{4}$
            \end{enumerate}
          \end{column}
        \end{columns}
      \end{example-highlight}
    \end{column}
    \begin{column}{0.49\textwidth}
      \vspace*{-0.09cm}
      \begin{highlight}
        \textbf{Observação.}
        \begin{itemize}
          \item Não existe função cujo gráfico seja simétrico com respeito ao eixo $x$.
        \end{itemize}
      \end{highlight}
    \end{column}
  \end{columns}
\end{frame}

\begin{frame}
  \begin{columns}[onlytextwidth]
    \begin{column}{0.49\textwidth}\vspace*{-0.5cm}
      \begin{definition}[Função Par]
        Uma função $f$ é dita \textbf{função par} se
        \begin{equation*}
          f(-x) = f(x),
        \end{equation*}
        para todo $x\in\Domain{f}$.
      \end{definition}
      \begin{highlight}
        \textbf{Observação.}
        \begin{itemize}
          \item O gráfico de uma função par é simétrico com respeito ao eixo $x$.
        \end{itemize}
      \end{highlight}
      \vspace*{0.2cm}
      \begin{example-highlight}
        \begin{enumerate}
          \item $y=x^{2}$, $y=x^{4}$, $y=x^{6}$, $y=x^{8}$, $\dots$
          \item $y=\dfrac{1}{x^{2}}$, $y=\dfrac{1}{x^{4}}$, $y=\dfrac{1}{x^{6}}$, $y=\dfrac{1}{x^{8}}$, $\dots$
        \end{enumerate}
      \end{example-highlight}
    \end{column}
    \begin{column}<2>{0.49\textwidth}\vspace*{-0.5cm}
      \begin{definition}[Função Ímpar]
        Uma função $f$ é dita \textbf{função ímpar} se
        \begin{equation*}
          f(-x) = -f(x),
        \end{equation*}
        para todo $x\in\Domain{f}$.
      \end{definition}
      \begin{highlight}
        \textbf{Observação.}
        \begin{itemize}
          \item O gráfico de uma função ímpar é simétrico com respeito à origem.
        \end{itemize}
      \end{highlight}
      \vspace*{0.2cm}
      \begin{example-highlight}
        \begin{enumerate}
          \item $y=x$, $y=x^{3}$, $y=x^{5}$, $y=x^{7}$, $\dots$
          \item $y=\dfrac{1}{x}$, $y=\dfrac{1}{x^{3}}$, $y=\dfrac{1}{x^{5}}$, $y=\dfrac{1}{x^{7}}$, $\dots$
        \end{enumerate}
      \end{example-highlight}
    \end{column}
  \end{columns}
\end{frame}

