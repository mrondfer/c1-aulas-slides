\section{Funções Racionais}

\subsection{Conceito de Função Racional}

\begin{frame}
  \begin{definition}[Função Racional]
    Sejam $p$ e $q$ funções polinomiais. Chamamos de função racional o quociente
    \begin{equation*}
      f(x) = \frac{p(x)}{q(x)}
    \end{equation*}
    definida para todo $x$ que esteja nos domínios de $p$ e $q$, exceto onde $q(x)=0$.
  \end{definition}
  \begin{highlight}
    \textbf{Alguns casos importantes:}
    \begin{enumerate}
      \item Se $\Domain{p}=\mathbb{R}$ e $\Domain{q}=\mathbb{R}$, então $\Domain{f} = \left\{x\in\mathbb{R}:q(x)\not=0\right\}$;
      \item Em geral, considerando restrições nos domínios das funções $p$ e $q$, podemos assumir que $\Domain{f}=\Domain{p}\cap\Domain{q}-\left\{x\in\mathbb{R}:q(x)=0\right\}$;
      \item Quando $p(x)=1$ para todo $x\in\mathbb{R}$, temos a família de \textbf{funções recíprocas}
      \begin{equation*}
        f(x) = \frac{1}{q(x)}
      \end{equation*}
    \end{enumerate}
  \end{highlight}
\end{frame}

\begin{frame}
  \begin{columns}[onlytextwidth]
    \begin{column}{0.49\textwidth}
      \vspace*{-0.35cm}
      \begin{example}
        Considere a família de funções racionais $f(x)=\dfrac{1}{x^{n}}$, com $n$ par.
      \end{example}
      \textbf{Observe o gráfico das funções}:
      \vspace*{-0.5cm}
      \begin{columns}[onlytextwidth]
        \begin{column}{0.5\textwidth}
          \begin{enumerate}
            \item< 1- > $f(x)=\dfrac{1}{x^{2}}$
            \item< 2- > $f(x)=\dfrac{1}{x^{4}}$
          \end{enumerate}
        \end{column}
        \begin{column}{0.5\textwidth}
          \begin{enumerate}
            \setcounter{enumi}{2}
            \item< 3- > $f(x)=\dfrac{1}{x^{6}}$
            \item< 4- > $f(x)=\dfrac{1}{x^{8}}$
          \end{enumerate}
        \end{column}
      \end{columns}
      \vspace*{0.35cm}
      \begin{highlight}
        \textbf{Note que:}
        \begin{itemize}
          \item< 1- > $f(1) = 1$ e $f(-1)=1$;
          \item< 4- > Se $|x| > 1$ e $p>q$, então $x^{p} > x^{q}$;
          \item< 4- > Se $|x| < 1$ e $p>q$, então $x^{p} < x^{q}$;
          \item< 4- > $\Domain{f}=\mathbb{R}^{*}$;
          \item< 4- > $\Image{f}=\mathbb{R}^{*}_{+} = \left\{x\in\mathbb{R}:x > 0\right\}$.
        \end{itemize}
      \end{highlight}
    \end{column}
    \begin{column}{0.49\textwidth}
        \begin{figure}
        \includefigure<1>[width=\textwidth]{umxn-par-1.pdf}
        \includefigure<2>[width=\textwidth]{umxn-par-2.pdf}
        \includefigure<3>[width=\textwidth]{umxn-par-3.pdf}
        \includefigure<4>[width=\textwidth]{umxn-par-4.pdf}
        \includefigure<5>[width=\textwidth]{umxn-par-5.pdf}
      \end{figure}
    \end{column}
  \end{columns}
\end{frame}

\begin{frame}
  \begin{columns}[onlytextwidth]
    \begin{column}{0.49\textwidth}
      \vspace*{-0.35cm}
      \begin{example}
        Considere a família de funções racionais $f(x)=\dfrac{1}{x^{n}}$, com $n$ ímpar.
      \end{example}
      \textbf{Observe o gráfico das funções}:
      \vspace*{-0.5cm}
      \begin{columns}[onlytextwidth]
        \begin{column}{0.5\textwidth}
          \begin{enumerate}
            \item< 1- > $f(x)=\dfrac{1}{x}$
            \item< 2- > $f(x)=\dfrac{1}{x^{3}}$
          \end{enumerate}
        \end{column}
        \begin{column}{0.5\textwidth}
          \begin{enumerate}
            \setcounter{enumi}{2}
            \item< 3- > $f(x)=\dfrac{1}{x^{5}}$
            \item< 4- > $f(x)=\dfrac{1}{x^{7}}$
          \end{enumerate}
        \end{column}
      \end{columns}
      \vspace*{0.35cm}
      \begin{highlight}
        \textbf{Note que:}
        \begin{itemize}
          \item< 1- > $f(1) = 1$ e $f(-1)=1$;
          \item< 4- > Se $|x| > 1$ e $p>q$, então $x^{p} > x^{q}$;
          \item< 4- > Se $|x| < 1$ e $p>q$, então $x^{p} < x^{q}$;
          \item< 4- > $\Domain{f}=\mathbb{R}^{*}$;
          \item< 4- > $\Image{f}=\mathbb{R}^{*} = \left\{x\in\mathbb{R}:x \not= 0\right\}$.
        \end{itemize}
      \end{highlight}
    \end{column}
    \begin{column}{0.49\textwidth}
        \begin{figure}
        \includefigure<1>[width=\textwidth]{umxn-impar-1.pdf}
        \includefigure<2>[width=\textwidth]{umxn-impar-2.pdf}
        \includefigure<3>[width=\textwidth]{umxn-impar-3.pdf}
        \includefigure<4>[width=\textwidth]{umxn-impar-4.pdf}
        \includefigure<5>[width=\textwidth]{umxn-impar-5.pdf}
        \includefigure<6>[width=\textwidth]{umxn-impar-6.pdf}
      \end{figure}
    \end{column}
  \end{columns}
\end{frame}