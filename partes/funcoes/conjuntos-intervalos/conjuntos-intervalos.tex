\section{Conjuntos Numéricos e Intervalos}
\begin{frame}
  \frametitle{Conjuntos Numéricos}
  \begin{itemize}
    \item Conjunto dos números \textbf{naturais}
    \begin{equation*}
      \mathbb{N} = \{1,\,2,\,3,\,4,\,5,\,\dots\}\quad\mbox{\textbf{ou}}\quad \mathbb{N} = \{0,\,1,\,2,\,3,\,4,\,5,\,\dots\}
    \end{equation*} 
    Vamos assumir que $0\in\mathbb{N}$!
    \item Conjunto dos números \textbf{inteiros}
    \begin{equation*}
      \mathbb{Z} = \{\dots,\,-5,\,-5,\,-3,\,-2,\,-1,\,0,\,1,\,2,\,3,\,4,\,5,\,\dots\}
    \end{equation*}
    \item Conjunto dos números \textbf{racionais}
    \begin{equation*}
      \mathbb{Q} = \left\{x\bigl| x=\frac{p}{q}, p\in\mathbb{Z}, q\in\mathbb{Z}^{*}\right\}
    \end{equation*}
    \item Conjunto dos números \textbf{reais}
    \begin{equation*}
      \mathbb{R} = \mathbb{Q}\cup \{\mbox{números irracionais}\}
    \end{equation*}
  \end{itemize}
\end{frame}

\section{Desigualdades, Intervalos e Valor Absoluto}

\begin{frame}
  \frametitle{Desiguadades e Intervalos}
  \begin{itemize}
    \item Intervalo \textbf{fechado}
    \begin{equation*}
      [a,\,b] = \left\{x\in\mathbb{R} | a \leq x \leq b\right\}
    \end{equation*}
    \item Intervalo \textbf{aberto}
    \begin{equation*}
      (a,\,b) = \left\{x\in\mathbb{R} | a < x < b\right\}
    \end{equation*}

    \item Intervalo \textbf{aberto à esquerda} e \textbf{fechado à direita}
    \begin{equation*}
      (a,\,b] = \left\{x\in\mathbb{R} | a < x \leq b\right\}
    \end{equation*}

    \item Intervalo \textbf{fechado à esquerda} e \textbf{aberto à direita}
    \begin{equation*}
      [a,\,b) = \left\{x\in\mathbb{R} | a \leq x < b\right\}
    \end{equation*}

    \item Como intervalos são conjuntos, temos também as clássicas operações de união ($A\cup B$), interseção ($A\cap B$) e diferença ($A-B$)
  \end{itemize}
\end{frame}

\begin{frame}
  \frametitle{Valor Absoluto}
  \begin{itemize}
    \item O valor absoluto de um número real $x$ é definido pela expressão
    \begin{equation*}
      |x| = \begin{cases}
        \phantom{-}x, & \mbox{se } x \geq 0 \\
        -x, & \mbox{se } x < 0
      \end{cases}
    \end{equation*}
    \item Geometricamente, representa a distância de $x$ à origem ($x=0$)
    \item Importante reconhecer algumas desigualdades que envolvem o valor absoluto
    \begin{equation*}
      |x-a| < K \Leftrightarrow a-K<x<a+K
    \end{equation*}
    \begin{equation*}
      |x-a| \leq K \Leftrightarrow a-K\leq x\leq a+K
    \end{equation*}
    \begin{equation*}
      |x-a| > K \Leftrightarrow x < a-K \mbox{ ou } x > a + K
    \end{equation*}
    \begin{equation*}
      |x-a| \geq K \Leftrightarrow x \geq a-K \mbox{ ou } x \geq a + K
    \end{equation*}
  \end{itemize}
\end{frame}
