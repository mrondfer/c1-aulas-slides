\section{Função Quadrática}

\subsection{Conceito de Função Quadrática}

\begin{frame}
  \begin{definition}[Função Quadrática]
    Chamamos de \textbf{função polinomial de 2º grau} ou \textbf{função quadrática} qualquer função \\ $f:\mathbb{R}\rightarrow\mathbb{R}$ tal que
    \begin{equation*}
      f(x) = ax^{2} + bx + c,
    \end{equation*}
    onde $a,\,b,\,c\in\mathbb{R}$ e $a\not=0$.
  \end{definition}
  \begin{columns}[onlytextwidth]
    \begin{column}{0.59\textwidth}
      \begin{highlight}
        \begin{itemize}
          \item Conforme a definição, $\Domain{f} = \mathbb{R}$, no entanto,
          nas aplicações, é comum considerar \emph{domínios menores}.
        \end{itemize}
      \end{highlight}
    \end{column}
    \begin{column}{0.39\textwidth}
      \begin{example-highlight}
        \begin{enumerate}
          \item $y = 5x^2 + 3x - 3$
          \item $f(x) = -x^{2} + 7x$
          \item $y = x^{2} - 9$
          \item $h(t) = -t^{2} + 5t - 8$
        \end{enumerate}
      \end{example-highlight}
    \end{column}
  \end{columns}
\end{frame}

% \begin{frame}
%   \begin{example}
%     Vamos iniciar considerando a mais básica de todas as funções quadráticas, a função $f(x) = x^{2}$, com o gráfico representado abaixo.
%   \end{example}
%   \begin{columns}[onlytextwidth]
%     \begin{column}{0.49\textwidth}
%       \begin{figure}
%         \includefigure{x2-exemplo.pdf}
%       \end{figure}
%     \end{column}
%     \begin{column}{0.49\textwidth}
%       \begin{highlight}
%         \textbf{Determine}
%         \begin{enumerate}
%           \item Domínio e imagem da função $f$\vspace{1.0cm}
%           \item $f(7)-f(-4)$\vspace{1.5cm}
%           \item Os valores de $x$ tais que $y = 36$\vspace*{1.0cm}
%         \end{enumerate}
%       \end{highlight}
%     \end{column}
%   \end{columns}
% \end{frame}

\subsection{Gráfico da Função Quadrática}

\begin{frame}
  \frametitle{Gráfico da Função Quadrática}
  \vspace*{-0.69cm}
  \begin{columns}[onlytextwidth]
    \begin{column}{0.74\textwidth}
      \begin{itemize}
        \item< 1- > Considere a função quadrática
        \begin{equation*}
          f(x) = ax^{2} + bx + c
        \end{equation*}
        \item< 2- > O gráfico é uma \textbf{parábola} com \emph{eixo de simetria} paralelo ao eixo $Oy$;
        \item< 3- > A \emph{concavidade} da parábola é voltada para:
        \begin{itemize}
          \item< 3- > \textbf{cima} quando $a > 0$;
          \item< 4- > \textbf{baixo} quando $a < 0$.
        \end{itemize}
        \item< 5- > As coordenadas do \emph{vértice} da parábola são dadas por
        \begin{equation*}
          V = \left(x_{v},\,y_{v}\right) = \left( -\frac{b}{2a},\,-\frac{\Delta}{4a} \right)
        \end{equation*}
        com $\Delta = b^{2} - 4ac$, o famoso discriminante;
        \item< 7- > O gráfico \emph{corta} o eixo $Oy$ no ponto $C(0,\,c)$;
        \item< 9- > O sinal de $b$ indica \emph{como o gráfico corta} o eixo $Oy$:
        \begin{itemize}
          \item< 9- > \textbf{crescente} quando $b > 0$;
          \item< 11- > \textbf{decrescente} quando $b < 0$;
          \item< 13- > \textbf{vértice no eixo} $Oy$ quando $b = 0$.
        \end{itemize}
      \end{itemize}
    \end{column}
    \begin{column}{0.26\textwidth}
      \begin{figure}
        \includefigure<2>[width=1\textwidth]{x2-grafico-1.pdf}
        \includefigure<3>[width=1\textwidth]{x2-grafico-2.pdf}
        \includefigure<4>[width=1\textwidth]{x2-grafico-3.pdf}
        \includefigure<5>[width=1\textwidth]{x2-grafico-4.pdf}
        \includefigure<6>[width=1\textwidth]{x2-grafico-5.pdf}
        \includefigure<7>[width=1\textwidth]{x2-grafico-6.pdf}
        \includefigure<8>[width=1\textwidth]{x2-grafico-7.pdf}
        \includefigure<9>[width=1\textwidth]{x2-grafico-8.pdf}
        \includefigure<10>[width=1\textwidth]{x2-grafico-9.pdf}
        \includefigure<11>[width=1\textwidth]{x2-grafico-10.pdf}
        \includefigure<12>[width=1\textwidth]{x2-grafico-11.pdf}
        \includefigure<13>[width=1\textwidth]{x2-grafico-12.pdf}
        \includefigure<14>[width=1\textwidth]{x2-grafico-13.pdf}
      \end{figure}
    \end{column}
  \end{columns}
\end{frame}

\subsection{Zeros da Função Quadrática}
\begin{frame}
  \frametitle{Zeros da Função Quadrática}
  \vspace*{-0.6cm}
  \begin{columns}[onlytextwidth]
    \begin{column}{0.6\textwidth}
      \begin{itemize}
        \item Considere a função quadrática
        \begin{equation*}
          f(x) = ax^{2} + bx + c
        \end{equation*}
        \item Os seus \emph{zeros} podem ser determinados ao calcular:
        \begin{equation*}
          f(x) = 0 \qquad\Longleftrightarrow\qquad ax^{2} + bx + c = 0
        \end{equation*}
        \item O que é feito pela aplicação da \textbf{fórmula}:
        \begin{equation*}
          x = \frac{-b\pm\sqrt{\Delta}}{2a}
        \end{equation*}
        onde $\Delta = b^{2} - 4ac$;
        \item< 2- > Existem \textbf{três casos} dependentes do sinal de $\Delta$:
        \begin{itemize}
          \item< 2- > $\Delta > 0$ -- dois zeros reais e \emph{distintos};
          \item< 4- > $\Delta = 0$ -- um zero \emph{duplo};
          \item< 6- > $\Delta < 0$ -- nenhum zero \emph{real}.
        \end{itemize}
      \end{itemize}
    \end{column}
    \begin{column}{0.4\textwidth}
      \vspace*{1cm}
      \begin{figure}
        \includefigure<2>[width=1\textwidth]{x2-grafico-14.pdf}
        \includefigure<3>[width=1\textwidth]{x2-grafico-15.pdf}
        \includefigure<4>[width=1\textwidth]{x2-grafico-16.pdf}
        \includefigure<5>[width=1\textwidth]{x2-grafico-17.pdf}
        \includefigure<6>[width=1\textwidth]{x2-grafico-18.pdf}
        \includefigure<7>[width=1\textwidth]{x2-grafico-19.pdf}
      \end{figure}
    \end{column}
  \end{columns}
\end{frame}

\subsection{Formas da Função Quadrática}
\begin{frame}
  \frametitle{Formas da Função Quadrática}
  \begin{itemize}
    \item Até o momento estávamos escrevendo a forma quadrática por meio de sua \textbf{forma geral}
    \begin{equation*}
      f(x) = ax^{2} + bx + c
    \end{equation*}
    \item No entanto, ao analisar uma função quadrática, dependendo dos objetivos da analise, ou das informações disponíveis, é conveniente escrever a função quadrática em \emph{outras formas};
  \end{itemize}
  \begin{columns}[onlytextwidth]
    \begin{column}{0.49\textwidth}
      \begin{overlayarea}{\textwidth}{0.6\textheight}
        \only<2>{
          \begin{itemize}
            \item Se conhecidos os zeros $x_{1}$ e $x_{2}$ da função quadrática, é possível escrevê-la em sua \textbf{forma fatorada}:
            \begin{equation*}
              f(x) = a(x-x_{1})(x-x_{2})
            \end{equation*}
            \item Expõe diretamente os zeros da função quadrática;
            \item Ainda é preciso conhecer ao menos um ponto adicional para que possamos determinar $a$.
          \end{itemize}
        }
        \only<3>{
          \begin{figure}
            \includefigure[width=0.65\textwidth]{x2-grafico-21.pdf}
          \end{figure}
        }
      \end{overlayarea}
    \end{column}
    \begin{column}{0.49\textwidth}
      \begin{overlayarea}{\textwidth}{0.6\textheight}
        \only<2>{
          \begin{figure}
            \includefigure[width=0.65\textwidth]{x2-grafico-20.pdf}
          \end{figure}
        }
        \only<3>{
          \begin{itemize}
            \item Se conhecido o vértice $V(x_{v},y_{v})$ do gráfico da função quadrática, é possível escrevê-la em sua \textbf{forma canônica}:
            \begin{equation*}
              f(x) = a(x-x_{v})^{2}+y_{v}
            \end{equation*}
            \item Expõe diretamente o vértice da função quadrática;
            \item Ainda é preciso conhecer ao menos um ponto adicional para que possamos determinar $a$.
          \end{itemize}
        }
      \end{overlayarea}
    \end{column}
  \end{columns}
\end{frame}

% \subsection{Exemplos}

% \begin{frame}\vspace{-0.5cm}
%   \begin{block}{Exemplo 3}
%     Sobre a função quadrática $f(x) = x^{2} - 6x + 5$
%   \end{block}
%   \begin{columns}[onlytextwidth]
%     \begin{column}{0.4\textwidth}
%       \textbf{Determine:}
%       \begin{enumerate}
%         \item<only@1> O vértice da parábola\vspace{1.5cm}
%         \item<only@1> Os zeros da função\vspace{2.0cm}
%         \item<only@1> O gráfico de $f$
%         \item<only@2> O domínio $D(f)$ e a imagem $Im(f)$\vspace{1.0cm}
%         \item<only@2> Qual a solução da inequação $x^{2} - 6x + 5 < 0$?
%       \end{enumerate}
%     \end{column}
%     \begin{column}{0.6\textwidth}
%       \begin{figure}
%         \vspace{-0.4cm}
%         \includefigure<1>[width=0.8\textwidth]{fig16.png}
%         \includefigure<2>[width=0.8\textwidth]{fig17.png}
%       \end{figure}
%     \end{column}
%   \end{columns}
% \end{frame}

% \begin{frame}\vspace{-0.5cm}
%   \begin{block}{Exemplo 4}
%     Sobre a função quadrática $f(x) = -x^{2} + 4x - 3$.
%   \end{block}
%   \begin{columns}[onlytextwidth]
%     \begin{column}{0.35\textwidth}
%       \textbf{Determine}
%       \begin{enumerate}
%         \item<only@1> Os zeros da função\vspace{1.5cm}
%         \item<only@1> O vértice da parábola\vspace{2.0cm}
%         \item<only@1> O gráfico de $f$
%         \item<only@2> Os conjuntos $D(f)$ e $Im(f)$\vspace{1.0cm}
%         \item<only@2> Estudo da monotonia\vspace{2.0cm}
%         \item<only@2> Estudo do sinal
%       \end{enumerate}
%     \end{column}
%     \begin{column}{0.6\textwidth}
%       \begin{figure}
%         \vspace{-0.7cm}
%         \includefigure<1>[width=0.8\textwidth]{fig18.png}
%         \includefigure<2>[width=0.8\textwidth]{fig19.png}
%       \end{figure}
%     \end{column}
%   \end{columns}
% \end{frame}

% \begin{frame}\vspace{-0.5cm}
%   \begin{block}{Exemplo 5}
%     Considere o gráfico da função quadrática abaixo.
%   \end{block}
%   \begin{columns}[onlytextwidth]
%     \begin{column}{0.4\textwidth}
%       \textbf{Determine}
%       \begin{enumerate}
%         \item<only@1> Intervalo de crescimento e decrescimento\vspace{1.0cm}
%         \item<only@1> Os valores de $x$ tais que $f(x) = 12$\vspace{1.5cm}
%         \item<only@1> Os valores de $x$ tais que $f(x) > 0$
%         \item<only@2> Sabendo que trata-se de uma função quadrática, determine a sua lei de formação
%       \end{enumerate}
%     \end{column}
%     \begin{column}{0.6\textwidth}
%       \begin{figure}
%         \vspace{-0.7cm}
%         \includefigure[width=0.8\textwidth]{fig20.png}
%       \end{figure}
%     \end{column}
%   \end{columns}
% \end{frame}

\section{Funções Polinomiais}

\subsection{Conceito de Função Polinomial}

\begin{frame}
  \begin{definition}[Função Polinomial]
    Chamamos de \textbf{função polinomial} qualquer função $f:\mathbb{R}\rightarrow\mathbb{R}$ tal que
    \begin{equation*}
      f(x) = a_{n}x^{n} + a_{n-1}x^{n-1} + \cdots a_{2}x^{2} + a_{1}x + a_{0},
    \end{equation*}
    onde $a_{n},a_{n-1},\dots,a_{2},a_{1},a_{0}\in\mathbb{R}$ e $a_{n}\not=0$. Dizemos que o \textbf{grau} da função polinomial é dado pelo valor de $n$.
  \end{definition}
  \begin{highlight}
    \textbf{Alguns casos importantes:}
    \begin{enumerate}
      \item Conforme a definição, $\Domain{f}=\mathbb{R}$, no entanto, nas aplicações, é comum considerar \emph{domínios menores};
      \item Se $n=1$, temos a função polinomial do 1º grau: $f(x) = ax + b$;
      \item Se $n=2$, temos a função polinomial do 2º grau: $f(x) = ax^{2} + bx + c$;
      \item Para $n > 2$, o comportamento pode ser estudado por meio do
      uso de \emph{limites} e \emph{derivadas}.
    \end{enumerate}
  \end{highlight}
\end{frame}

\begin{frame}
  \begin{columns}[onlytextwidth]
    \begin{column}{0.49\textwidth}
      \vspace*{-0.35cm}
      \begin{example}
        Considere a família de funções polinomiais $f(x)=x^{n}$, com $n$ par.
      \end{example}
      \textbf{Observe o gráfico das funções}:
      \vspace*{-0.5cm}
      \begin{columns}[onlytextwidth]
        \begin{column}{0.5\textwidth}
          \begin{enumerate}
            \item< 1- > $f(x)=x^{2}$
            \item< 2- > $f(x)=x^{4}$
          \end{enumerate}
        \end{column}
        \begin{column}{0.5\textwidth}
          \begin{enumerate}
            \setcounter{enumi}{2}
            \item< 3- > $f(x)=x^{6}$
            \item< 4- > $f(x)=x^{8}$
          \end{enumerate}
        \end{column}
      \end{columns}
      \vspace*{0.35cm}
      \begin{highlight}
        \textbf{Note que:}
        \begin{itemize}
          \item< 1- > $f(0)=0$;
          \item< 1- > $f(1) = 1$ e $f(-1)=1$;
          \item< 4- > Se $|x| > 1$ e $p>q$, então $x^{p} > x^{q}$;
          \item< 4- > Se $|x| < 1$ e $p>q$, então $x^{p} < x^{q}$;
          \item< 4- > $\Domain{f}=\mathbb{R}$;
          \item< 4- > $\Image{f}=\mathbb{R}_{+} = \left\{x\in\mathbb{R}:x\geq 0\right\}$.
        \end{itemize}
      \end{highlight}
    \end{column}
    \begin{column}{0.49\textwidth}
        \begin{figure}
        \includefigure<1>[width=\textwidth]{xn-par-1.pdf}
        \includefigure<2>[width=\textwidth]{xn-par-2.pdf}
        \includefigure<3>[width=\textwidth]{xn-par-3.pdf}
        \includefigure<4>[width=\textwidth]{xn-par-4.pdf}
        \includefigure<5>[width=\textwidth]{xn-par-5.pdf}
      \end{figure}
    \end{column}
  \end{columns}
\end{frame}

\begin{frame}
  \begin{columns}[onlytextwidth]
    \begin{column}{0.49\textwidth}
      \vspace*{-0.35cm}
      \begin{example}
        Considere a família de funções polinomiais $f(x)=x^{n}$, com $n$ ímpar.
      \end{example}
      \textbf{Observe o gráfico das funções}:
      \vspace*{-0.5cm}
      \begin{columns}[onlytextwidth]
        \begin{column}{0.5\textwidth}
          \begin{enumerate}
            \item< 1- > $f(x)=x$
            \item< 2- > $f(x)=x^{3}$
          \end{enumerate}
        \end{column}
        \begin{column}{0.5\textwidth}
          \begin{enumerate}
            \setcounter{enumi}{2}
            \item< 3- > $f(x)=x^{5}$
            \item< 4- > $f(x)=x^{7}$
          \end{enumerate}
        \end{column}
      \end{columns}
      \vspace*{0.35cm}
      \begin{highlight}
        \textbf{Note que:}
        \begin{itemize}
          \item< 1- > $f(0)=0$;
          \item< 1- > $f(1) = 1$ e $f(-1)=-1$;
          \item< 4- > Se $|x| > 1$ e $p>q$, então $|x^{p}| > |x^{q}|$;
          \item< 4- > Se $|x| < 1$ e $p>q$, então $|x^{p}| < |x^{q}|$;
          \item< 4- > $\Domain{f}=\mathbb{R}$;
          \item< 4- > $\Image{f}=\mathbb{R}$.
        \end{itemize}
      \end{highlight}
    \end{column}
    \begin{column}{0.49\textwidth}
        \begin{figure}
        \includefigure<1>[width=\textwidth]{xn-impar-1.pdf}
        \includefigure<2>[width=\textwidth]{xn-impar-2.pdf}
        \includefigure<3>[width=\textwidth]{xn-impar-3.pdf}
        \includefigure<4>[width=\textwidth]{xn-impar-4.pdf}
        \includefigure<5>[width=\textwidth]{xn-impar-5.pdf}
      \end{figure}
    \end{column}
  \end{columns}
\end{frame}