\section{Funções Exponenciais}

\subsection{Função Exponencial}
\begin{frame}
  \begin{columns}[onlytextwidth]
    \begin{column}{0.49\textwidth}\vspace*{-0.5cm}
      \begin{definition}[Função Exponencial]
        Seja $a$ um número real, com $a > 0$ e $a\not= 1$. Chamamos de \textbf{função exponencial} de base $a$ a função $f:\mathbb{R}\rightarrow\mathbb{R}_{+}^{*}$ que associa a cada $x\in\mathbb{R}$ o número real\vspace*{-0.2cm}
        \begin{equation*}
          f(x) = a^{x}.\vspace*{-0.2cm}
        \end{equation*}
      \end{definition}
      \begin{example-highlight}
        \begin{enumerate}
          \item $y = 10^{x}$
          \item<2> $y = \left( \dfrac{1}{3} \right)^{x}$
        \end{enumerate}
      \end{example-highlight}
      \begin{highlight}
        \textbf{Em ambos os casos, temos que:}
        \begin{itemize}
          \item $\Domain{f} = \mathbb{R}$
          \item $\Image{f} = \mathbb{R}_{+}^{*}$
        \end{itemize}
      \end{highlight}
    \end{column}
    \begin{column}{0.49\textwidth}\vspace*{-0.8cm}
      \begin{figure}
        \includefigure<1>[width=\textwidth]{funcao-exponencial-1.pdf}
        \includefigure<2>[width=\textwidth]{funcao-exponencial-2.pdf}
      \end{figure}
    \end{column}
  \end{columns}
\end{frame}

\begin{frame}[c]
  \begin{figure}
    \includefigure[width=0.95\textwidth]{funcao-exponencial-bases-crescentes.pdf}
  \end{figure}
\end{frame}

\begin{frame}[c]
  \begin{figure}
    \includefigure[width=0.95\textwidth]{funcao-exponencial-bases-decrescentes.pdf}
  \end{figure}
\end{frame}