\section{Função Afim}

\begin{frame}
  \begin{definition}[Função Afim]
    Dados $a$ e $b$ números reais, chama-se de \textbf{função afim} qualquer função $f:\mathbb{R}\rightarrow\mathbb{R}$ tal que $f(x) = ax+b$.
  \end{definition}

  \textbf{Casos especiais}
  \begin{itemize}
    \item O número $a$ é chamado de \textbf{coeficiente angular}
    \item O número $b$ é chamado de coeficiente linear ou termo independente
    \item Se $a = 0$, então $f(x) = b$ e temos a chamada \textbf{função constante}.
    \item Se $a \not= 0$, temos a função polinomial do primeiro grau
    \item Se $b=0$, temos a \textbf{função linear}
    \item Se $a=1$ e $b=0$, temos a \textbf{função identidade}
  \end{itemize}

  \textbf{Gráfico:} é uma reta, basta conhecer dois pontos para conhecer todos.

  \textbf{Exemplo:} o problema da conta de água.
\end{frame}

\begin{frame}
  \begin{theorem}[Monotonia da Função Afim]
    Seja $f(x) = ax + b$ uma função afim. Então $f$ é:
    \begin{enumerate}
      \item monótona crescente para todo $x\in\mathbb{R}$ quando $a>0$;
      \item monótona constante para todo $x\in\mathbb{R}$ quando $a=0$;
      \item monótona decrescente para todo $x\in\mathbb{R}$ quando $a<0$;
    \end{enumerate}
  \end{theorem}
  \begin{figure}
    \includefigure[scale=1.2]{teorema-monotonia-crescente.pdf}
    \includefigure[scale=1.2]{teorema-monotonia-constante.pdf}
    \includefigure[scale=1.2]{teorema-monotonia-decrescente.pdf}
  \end{figure}
\end{frame}

\begin{frame}
  \begin{columns}[onlytextwidth]
    \begin{column}{0.49\textwidth}
      \begin{example}
        Considere a função $f(x) = 2x - 4$ de $\mathbb{R}$ em $\mathbb{R}$. Determine:
      \end{example}
      \begin{highlight}
        \begin{enumerate}
          \item $f(7)$ e $f(-4)$
          \item O valor de $x$ tal que $f(x) = 28$.
          \item O(s) zero(s) da função.
          \item O esboço do gráfico.
          \item $x$ tais que $f(x) > 0$
          \item $x$ tais que $f(x) < 0$
        \end{enumerate}
      \end{highlight}
    \end{column}

    \begin{column}{0.49\textwidth}
    \end{column}
  \end{columns}
\end{frame}

\begin{frame}
    \begin{columns}[onlytextwidth]
    \begin{column}{0.5\textwidth}
      \begin{example}
        Determine a lei da função representada no gráfico abaixo.
      \end{example}%
      \vspace*{-0.25cm}
      \begin{figure}
        \includefigure{exemplo-1-obter-lei.pdf}
      \end{figure}
    \end{column}
    \begin{column}{0.5\textwidth}
    \end{column}
  \end{columns}
\end{frame}