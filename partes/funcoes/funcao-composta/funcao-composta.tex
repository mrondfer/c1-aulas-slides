\section{Função Composta}

\begin{frame}
  \begin{definition}[Função Composta]
    Sejam $f:A\rightarrow B$ e $g:B\rightarrow C$ funções reais. Definimos a composição de $g$ e $f$ como a função $g\circ f:A\rightarrow C$ dada pela expressão
    \begin{equation*}
      (g\circ f)(x) = g(f(x)).
    \end{equation*}
  \end{definition}
  \begin{columns}[onlytextwidth]
    \begin{column}{0.49\textwidth}
      \begin{figure}
        \includefigure[width=\textwidth]{diagrama-venn-funcao-composta.pdf}
      \end{figure}
    \end{column}
    \begin{column}{0.49\textwidth}
      \begin{highlight}
        \begin{itemize}
          \item Lê-se: ``$g$ \emph{composta} $f$'' ou ``$g$ \emph{bola} $f$'';
          \item Primeiro calculamos $f(x)$; depois $g(f(x))$; 
          \item Nos referenciamos como ``função de dentro'', a $f$, e ``função de fora'', a $g$;
          \item Em geral, temos que $g\circ f \not= f\circ g$;
          \item Definida para três ou mais funções:\vspace*{-0.1cm}
          \begin{equation*}
            (h\circ g\circ f)(x) = h(g(f(x)))\vspace*{-0.1cm}
          \end{equation*}
        \end{itemize}
      \end{highlight}
    \end{column}
  \end{columns}
\end{frame}

\begin{frame}
  \begin{example}
    Dadas as funções $f(x) = \sqrt{x}$ e $g(x) = x^{2} + 1$, determine as composições a seguir, destacando os respectivos domínios.
    \begin{columns}[onlytextwidth]
        \begin{column}{0.2\textwidth}\vspace*{-0.4cm}
          \begin{enumerate}
            \item $g\circ f$
          \end{enumerate}
        \end{column}
        \begin{column}{0.2\textwidth}\vspace*{-0.4cm}
          \begin{enumerate}
            \setcounter{enumi}{1}
            \item $f\circ g$
          \end{enumerate}
        \end{column}
        \begin{column}{0.6\textwidth}\vspace*{-0.4cm}
        \end{column}
      \end{columns}
  \end{example}
\end{frame}

\begin{frame}
  \begin{example}
    Expresse as funções a seguir como uma composição de funções.
    \begin{columns}[onlytextwidth]
        \begin{column}{0.2\textwidth}\vspace*{-0.4cm}
          \begin{enumerate}
            \item $\tan{x^{5}}$
          \end{enumerate}
        \end{column}
        \begin{column}{0.2\textwidth}\vspace*{-0.4cm}
          \begin{enumerate}
            \setcounter{enumi}{1}
            \item $\tan^{2}{x}$
          \end{enumerate}
        \end{column}
        \begin{column}{0.2\textwidth}\vspace*{-0.4cm}
          \begin{enumerate}
            \setcounter{enumi}{2}
            \item $\tan^{2}{x^{5}}$
          \end{enumerate}
        \end{column}
        \begin{column}{0.4\textwidth}\vspace*{-0.4cm}
          \begin{enumerate}
            \setcounter{enumi}{3}
            \item $(x+1)^{6}=(x+1)^{2\cdot 3}$
          \end{enumerate}
        \end{column}
      \end{columns}
  \end{example}
\end{frame}
