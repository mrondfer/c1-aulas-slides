\section{Funções Trigonométricas Inversas}

\subsection{Arco Seno}
\begin{frame}
  \begin{columns}[onlytextwidth]
    \begin{column}{0.49\textwidth}\vspace{-0.5cm}
      \begin{highlight}
        \textbf{Definimos anteriormente:}

        A função seno como $f:\R\rightarrow\R$, que a cada $x\in\R$ associa o número real $f(x)=\sen{x}$.
      \end{highlight}
    \end{column}
    \begin{column}{0.49\textwidth}\vspace*{-0.5cm}
      \begin{highlight}
        \textbf{Temos que a função seno possui:}
        \begin{itemize}
          \item $\Domain{f} = \R$
          \item $\Image{f} = [-1,\,1]$
          \item $T = 2\pi$ rad
        \end{itemize}
      \end{highlight}
    \end{column}
  \end{columns}
  \begin{figure}
    \includefigure<1>[width=\textwidth]{funcao-seno-1.pdf}
    \includefigure<2>[width=\textwidth]{funcao-seno-2.pdf}
  \end{figure}
  \vspace*{-1.2em}
  \begin{itemize}
    \item A função seno, como definimos, \textbf{não} é injetora \textbf{nem} sobrejetora. Logo, \textbf{não} é bijetora e, portanto, \textbf{não admite} uma inversa;
    \item<2> Se restringirmos $f:\left[-\frac{\pi}{2},\frac{\pi}{2}\right]\rightarrow[-1,1]$, teremos uma função bijetora!
  \end{itemize}
\end{frame}

\begin{frame}
  \begin{columns}[onlytextwidth]
    \begin{column}{0.49\textwidth}\vspace{-0.5cm}
      \begin{definition}[Função Arco Seno]
        Dada a função $y=\sen{x}$, com $-\pi/2\leq x\leq \pi/2$ e $-1\leq y\leq 1$, definimos a função \textbf{arco seno}, denotada como
        \begin{equation*}
          f(x) = \arcsen{x},
        \end{equation*}
        como a inversa desta função seno.
      \end{definition}
      \begin{highlight}
        \textbf{Observações:}
        \begin{itemize}
          \item $\Domain{f} = [-1,1]$;
          \item $\Image{f} = \left[-\frac{\pi}{2},\frac{\pi}{2}\right]$;
          \item A função arco seno associa cada $x\in[-1,1]$ com \emph{um arco} que resultaria em tal valor de seno.
        \end{itemize}
      \end{highlight}
    \end{column}
    \begin{column}{0.49\textwidth}\vspace*{-0.5cm}
      \begin{figure}
        \includefigure<2>[width=\textwidth]{funcao-arco-seno-1.pdf}
        \includefigure<3>[width=\textwidth]{funcao-arco-seno-2.pdf}
        \includefigure<4>[width=\textwidth]{funcao-arco-seno-3.pdf}
        \includefigure<5>[width=\textwidth]{funcao-arco-seno-4.pdf}
        \includefigure<6>[width=\textwidth]{funcao-arco-seno-5.pdf}
      \end{figure}
    \end{column}
  \end{columns}
\end{frame}

\subsection{Arco Cosseno}
\begin{frame}
  \begin{columns}[onlytextwidth]
    \begin{column}{0.49\textwidth}\vspace{-0.5cm}
      \begin{highlight}
        \textbf{Definimos anteriormente:}

        A função cosseno como $f:\R\rightarrow\R$, que a cada $x\in\R$ associa o número real $f(x)=\cos{x}$.
      \end{highlight}
    \end{column}
    \begin{column}{0.49\textwidth}\vspace*{-0.5cm}
      \begin{highlight}
        \textbf{Temos que a função cosseno possui:}
        \begin{itemize}
          \item $\Domain{f} = \R$
          \item $\Image{f} = [-1,\,1]$
          \item $T = 2\pi$ rad
        \end{itemize}
      \end{highlight}
    \end{column}
  \end{columns}
  \begin{figure}
    \includefigure<1>[width=\textwidth]{funcao-cosseno-1.pdf}
    \includefigure<2>[width=\textwidth]{funcao-cosseno-2.pdf}
  \end{figure}
  \vspace*{-1.2em}
  \begin{itemize}
    \item A função cosseno, como definimos, \textbf{não} é injetora \textbf{nem} sobrejetora. Logo, \textbf{não} é bijetora e, portanto, \textbf{não admite} uma inversa;
    \item<2> Se restringirmos $f:[0,\pi]\rightarrow[-1,1]$, teremos uma função bijetora!
  \end{itemize}
\end{frame}

\begin{frame}
  \begin{columns}[onlytextwidth]
    \begin{column}{0.49\textwidth}\vspace{-0.5cm}
      \begin{definition}[Função Arco Cosseno]
        Dada a função $y=\cos{x}$, com $0\leq x\leq \pi$ e $-1\leq y\leq 1$, definimos a função \textbf{arco cosseno}, denotada como
        \begin{equation*}
          f(x) = \arccos{x},
        \end{equation*}
        como a inversa desta função cosseno.
      \end{definition}
      \begin{highlight}
        \textbf{Observações:}
        \begin{itemize}
          \item $\Domain{f} = [-1,1]$;
          \item $\Image{f} = [0,\pi]$;
          \item A função arco cosseno associa cada $x\in[-1,1]$ com \emph{um arco} que resultaria em tal valor de cosseno.
        \end{itemize}
      \end{highlight}
    \end{column}
    \begin{column}{0.49\textwidth}\vspace*{-0.5cm}
      \begin{figure}
        \includefigure<2>[width=\textwidth]{funcao-arco-cosseno-1.pdf}
        \includefigure<3>[width=\textwidth]{funcao-arco-cosseno-2.pdf}
        \includefigure<4>[width=\textwidth]{funcao-arco-cosseno-3.pdf}
        \includefigure<5>[width=\textwidth]{funcao-arco-cosseno-4.pdf}
        \includefigure<6>[width=\textwidth]{funcao-arco-cosseno-5.pdf}
      \end{figure}
    \end{column}
  \end{columns}
\end{frame}

\subsection{Arco Tangente}
\begin{frame}
  \begin{columns}[onlytextwidth]
    \begin{column}{0.49\textwidth}\vspace{-0.5cm}
      \begin{highlight}
        \textbf{Definimos anteriormente:}

        A função tangente como $f:A\rightarrow\R$, que a cada $x\in\R$ associa o número real $f(x)=\tan{x}$, onde\vspace*{-0.25cm}
        \begin{equation*}
          A=\left\{x\in\R : x\not= \frac{\pi}{2} + k\cdot\pi, k\in\mathbb{Z}\right\}\vspace*{-0.8cm}
        \end{equation*}
      \end{highlight}
    \end{column}
    \begin{column}{0.49\textwidth}\vspace*{-0.5cm}
      \begin{highlight}
        \textbf{Temos que a função tangente possui:}
        \begin{itemize}
          \item $\Domain{f} = A$
          \item $\Image{f} = \R$
          \item $T = \pi$ rad
        \end{itemize}
      \end{highlight}
    \end{column}
  \end{columns}
  \begin{figure}
    \includefigure<1>[width=\textwidth]{funcao-tangente-1.pdf}
    \includefigure<2>[width=\textwidth]{funcao-tangente-2.pdf}
  \end{figure}
  \vspace*{-1.75em}
  \begin{itemize}
    \item A função tangente, como definimos, \textbf{não} é injetora. Logo, \textbf{não} é bijetora e, portanto, \textbf{não admite} uma inversa;
    \item<2> Se restringirmos $f:\left(-\frac{\pi}{2},\frac{\pi}{2}\right)\rightarrow\R$, teremos uma função bijetora!
  \end{itemize}
\end{frame}

\begin{frame}
  \begin{columns}[onlytextwidth]
    \begin{column}{0.49\textwidth}\vspace{-0.5cm}
      \begin{definition}[Função Arco Tangente]
        Dada a função $y=\tan{x}$, com $-\pi/2 < x < \pi/2$ e $y\in\R$, definimos a função \textbf{arco tangente}, denotada como
        \begin{equation*}
          f(x) = \arctan{x},
        \end{equation*}
        como a inversa desta função tangente.
      \end{definition}
      \begin{highlight}
        \textbf{Observações:}
        \vspace{-0.05cm}
        \begin{itemize}
          \item $\Domain{f} = \R$;
          \item $\Image{f} = \left(-\frac{\pi}{2},\frac{\pi}{2}\right)$;
          \item A função arco tangente associa cada $x\in\R$ com \emph{um arco} que resultaria em tal valor de tangente.
        \end{itemize}
      \end{highlight}
    \end{column}
    \begin{column}{0.49\textwidth}\vspace*{-0.5cm}
      \begin{figure}
        \includefigure<2>[width=\textwidth]{funcao-arco-tangente-1.pdf}
        \includefigure<3>[width=\textwidth]{funcao-arco-tangente-2.pdf}
        \includefigure<4>[width=\textwidth]{funcao-arco-tangente-3.pdf}
        \includefigure<5>[width=\textwidth]{funcao-arco-tangente-4.pdf}
        \includefigure<6>[width=\textwidth]{funcao-arco-tangente-5.pdf}
      \end{figure}
    \end{column}
  \end{columns}
\end{frame}

\subsection{Arco Secante}
\begin{frame}
  \begin{columns}[onlytextwidth]
    \begin{column}{0.49\textwidth}\vspace{-0.5cm}
      \begin{highlight}
        \textbf{Definimos anteriormente:}

        A função secante como $f:A\rightarrow\R$, que a cada $x\in\R$ associa o número real $f(x)=\sec{x}$, onde\vspace*{-0.2cm}
        \begin{equation*}
          A=\left\{x\in\R : x\not= \frac{\pi}{2} + k\cdot\pi, k\in\mathbb{Z}\right\}\vspace*{-0.8cm}
        \end{equation*}
      \end{highlight}
    \end{column}
    \begin{column}{0.49\textwidth}\vspace*{-0.5cm}
      \begin{highlight}
        \textbf{Temos que a função secante possui:}
        \begin{itemize}
          \item Domínio: $\Domain{f} = A$
          \item Imagem: $\Image{f} = (-\infty,-1]\cup[1,+\infty)$
          \item Período: $T = 2\pi$ rad
        \end{itemize}
      \end{highlight}
    \end{column}
  \end{columns}
  \vspace*{-0.5em}
  \begin{figure}
    \includefigure<1>[width=\textwidth]{funcao-secante-1.pdf}
    \includefigure<2>[width=\textwidth]{funcao-secante-2.pdf}
  \end{figure}
  \vspace*{-1.75em}
  \begin{itemize}
    \item A função secante, como definimos, \textbf{não} é injetora e \textbf{não} é sobrejetora. Logo, \textbf{não} é bijetora e, portanto, \textbf{não admite} uma inversa;
    \item<2> Se restringirmos $f:[0,\pi]-\left\{\frac{\pi}{2}\right\}\rightarrow(-\infty,-1]\cup[1,+\infty)$, teremos uma função bijetora!
  \end{itemize}
\end{frame}

\begin{frame}
  \begin{columns}[onlytextwidth]
    \begin{column}{0.49\textwidth}\vspace{-0.5cm}
      \begin{definition}[Função Arco Secante]
        Dada a função $y=\sec{x}$, com $0 < x < \pi$, $x\not=\pi/2$, e $y\in(-\infty,-1]\cup[1,+\infty)$, definimos a função \textbf{arco secante}, denotada como
        \begin{equation*}
          f(x) = \arcsec{x},
        \end{equation*}
        como a inversa desta função secante.
      \end{definition}
      \begin{highlight}
        \textbf{Observações:}
        \vspace{-0.05cm}
        \begin{itemize}
          \item $\Domain{f} = (-\infty,-1]\cup[1,+\infty)$;
          \item $\Image{f} = [0,\pi]-\left\{\frac{\pi}{2}\right\}$;
          \item A função arco secante associa cada $x\in\R$ com \emph{um arco} que resultaria em tal valor de secante.
        \end{itemize}
      \end{highlight}
    \end{column}
    \begin{column}{0.49\textwidth}\vspace*{-0.5cm}
      \begin{figure}
        \includefigure<2>[width=\textwidth]{funcao-arco-secante-1.pdf}
        \includefigure<3>[width=\textwidth]{funcao-arco-secante-2.pdf}
        \includefigure<4>[width=\textwidth]{funcao-arco-secante-3.pdf}
        \includefigure<5>[width=\textwidth]{funcao-arco-secante-4.pdf}
        \includefigure<6>[width=\textwidth]{funcao-arco-secante-5.pdf}
      \end{figure}
    \end{column}
  \end{columns}
\end{frame}
