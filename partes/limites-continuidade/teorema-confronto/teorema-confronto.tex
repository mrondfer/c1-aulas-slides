\section{Teorema do Confronto}

\begin{frame}
  \begin{theorem}[Teorema do Confronto (ou Teorema do Sanduíche)]
  Sejam $f$, $g$ e $h$ funções que satisfazem
  \begin{equation*}
    g(x) \leq f(x) \leq h(x)
  \end{equation*}
  para todo $x$ em um intervalo aberto que contenha o ponto $x_{0}$, exceto possivelmente em $x=x_{0}$. Se $g$ e $h$ possuírem o mesmo limite quando $x$ tende para $x_{0}$, isto é,
  \begin{equation*}
    \lim_{x\to x_{0}}g(x) = \lim_{x\to x_{0}}h(x) = L,
  \end{equation*}
  então $f$ também tem esse limite quando $x$ tende para $x_{0}$, ou seja,
  \begin{equation*}
    \lim_{x\to x_{0}}f(x) = L.
  \end{equation*}
  \end{theorem}
\end{frame}

\begin{frame}[c]
  \begin{figure}
    \includefigure[width=\textwidth]{exemplo1.pdf}
  \end{figure}
\end{frame}
