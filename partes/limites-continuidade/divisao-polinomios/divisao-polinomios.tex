\section{Divisão de Polinômios}

\begin{frame}
  \begin{theorem}[Teorema de D'Alembert]
    O resto $r(x)$ da divisão de um polinômio $f(x)$ por $x-a$ é $f(a)$. Assim o polinômio f(x) é divisível por $x-a$ se, e somente se, $f(a)=0$.
  \end{theorem}
  \vspace{0.1cm}
  \begin{columns}[onlytextwidth]
    \begin{column}{0.7\textwidth}
      \begin{example-highlight}
        Mostre que $f(x)=2x^{3}-3x^{2}-5x-12$ é divisível por $x-3$ utilizando o teorema de D’Alembert. 
      \end{example-highlight}
    \end{column}
    \begin{column}{0.3\textwidth}
    \end{column}
  \end{columns}
\end{frame}

\begin{frame}
  \begin{example}
    Verifique que $f(x)=2x^{3}-3x^{2}-5x-12$ é divisível por $x-3$ utilizando algoritmo da divisão polinomial.
  \end{example}
\end{frame}
