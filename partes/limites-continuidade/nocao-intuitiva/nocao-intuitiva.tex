\section{Noção Intuitiva de Limite}

\begin{frame}
  \begin{columns}[onlytextwidth]
    \begin{column}{0.75\textwidth}\vspace{-0.5cm}
      \begin{example}
        Estude os valores de $y = f(x) = 2x-1$ \emph{no entorno} de $x=2$.
      \end{example}
      \only<2>{
        \begin{itemize}\small
          \item Há duas formas para nos aproximarmos de $x=2$: pela \textbf{esquerda} (valores \textbf{menores} que $2$); ou pela \textbf{direita} (valores \textbf{maiores} que $2$).\vspace{0.2cm}
        \end{itemize}
        \begin{columns}[onlytextwidth]
          \begin{column}{0.49\textwidth}\small\centering
            \textbf{Limite lateral pela Esquerda}
            \begin{itemize}
              \item A medida em que $x\to 2^{-}$, temos que $y\to 3$
              \item É denotado por\vspace{-0.2cm}
              \begin{equation*}
                \lim_{x\to 2^{-}}{(2x-1)} = 3
              \end{equation*}
            \end{itemize}
          \end{column}
          \begin{column}{0.49\textwidth}\small\centering
            \textbf{Limite lateral pela Direita}
            \begin{itemize}
              \item A medida em que $x\to 2^{+}$, temos que $y\to 3$
              \item É denotado por\vspace{-0.2cm}
              \begin{equation*}
                \lim_{x\to 2^{+}}{(2x-1)} = 3
              \end{equation*}
            \end{itemize}
          \end{column}
        \end{columns}
        \begin{itemize}\small
          \item Quando os limites laterais são idênticos, temos que \textbf{existe} o \textbf{limite} \vspace{-0.2cm}
          \begin{equation*}
            \lim_{x\to 2^{-}}{(2x-1)} =  \lim_{x\to 2^{+}}{(2x-1)} =  \lim_{x\to 2}{(2x-1)} = 3\vspace{-0.2cm}
          \end{equation*}
          \item Note que o limite é o próprio valor da função em $x=2$\vspace{-0.2cm}
          \begin{equation*}
            f(2) = \lim_{x\to 2}{(2x-1)} = 3
          \end{equation*}
        \end{itemize}
      }
    \end{column}
    \begin{column}<2>{0.25\textwidth}\vspace{-0.6cm}
      \begin{table}[]
        \centering
        \begin{tabular}{
        >{\columncolor[HTML]{359830}}c 
        >{\columncolor[HTML]{F3F2F0}}c }
        \multicolumn{2}{c}{\cellcolor[HTML]{359830}{\color[HTML]{F3F2F0} \textbf{$x\to 2^{-}$}}} \\
        {\color[HTML]{F3F2F0} \textbf{$x$}}      & \textbf{$y$} \\
        {\color[HTML]{FFFEFE} \textbf{$1$}}      & $1$          \\
        {\color[HTML]{FFFEFE} \textbf{$1,5$}}    & $2$          \\
        {\color[HTML]{FFFEFE} \textbf{$1,9$}}    & $2,8$        \\
        {\color[HTML]{FFFEFE} \textbf{$1,99$}}   & $2,98$       \\
        {\color[HTML]{FFFEFE} \textbf{$1,999$}}  & $2,998$      \\
        {\color[HTML]{FFFEFE} \textbf{$1,9999$}} & $2,9998$    
        \end{tabular}%
        \end{table}
        \vspace{-0.2cm}
        \begin{table}[]
          \centering
          \begin{tabular}{
          >{\columncolor[HTML]{359830}}c 
          >{\columncolor[HTML]{F3F2F0}}c }
          \multicolumn{2}{c}{\cellcolor[HTML]{359830}{\color[HTML]{F3F2F0} \textbf{$x\to 2^{+}$}}} \\
          {\color[HTML]{F3F2F0} \textbf{$x$}}      & \textbf{$y$} \\
          {\color[HTML]{FFFEFE} \textbf{$3$}}      & $5$          \\
          {\color[HTML]{FFFEFE} \textbf{$2,5$}}    & $4$          \\
          {\color[HTML]{FFFEFE} \textbf{$2,1$}}    & $3,2$        \\
          {\color[HTML]{FFFEFE} \textbf{$2,01$}}   & $3,02$       \\
          {\color[HTML]{FFFEFE} \textbf{$2,001$}}  & $3,002$      \\
          {\color[HTML]{FFFEFE} \textbf{$2,0001$}} & $3,0002$    
          \end{tabular}%
          \end{table}
    \end{column}
  \end{columns}
\end{frame}

\begin{frame}
  \begin{columns}[onlytextwidth]
    \begin{column}{0.75\textwidth}\vspace{-0.55cm}
      \begin{example}
        Calcular o limite solicitado.
      \end{example}
      \begin{enumerate}\small
        \item $\displaystyle\lim_{x\to 1}{\frac{x^{2}+x-2}{x-1}}$\vspace{0.3cm}
      \end{enumerate}
      \only<2>{
        \begin{columns}[onlytextwidth]
          \begin{column}{0.49\textwidth}\small\centering
            \textbf{Limite Lateral pela Esquerda}
            \begin{equation*}
              \lim_{x\to 1^{-}}{\frac{x^{2}+x-2}{x-1}} = 3
            \end{equation*}
          \end{column}
          \begin{column}{0.49\textwidth}\small\centering
            \textbf{Limite Lateral pela Direita}
            \begin{equation*}
              \lim_{x\to 1^{+}}{\frac{x^{2}+x-2}{x-1}} = 3
            \end{equation*}
          \end{column}
        \end{columns}
        \begin{itemize}\small
          \item Como os limites laterais são idênticos, temos que
          \begin{equation*}
            \lim_{x\to 1}{\frac{x^{2}+x-2}{x-1}} = 3
          \end{equation*}
          \item Note que não é possível calcularmos $\dfrac{x^{2}+x-2}{x-1}$ quando $x = 1$
          \item Em geral, não é verdade que o limite seja o próprio valor da função\textbf{!}
        \end{itemize}
      }
    \end{column}
    \begin{column}<2>{0.25\textwidth}\vspace{-0.85cm}
      \begin{table}[]
        \centering
        \begin{tabular}{
        >{\columncolor[HTML]{359830}}c 
        >{\columncolor[HTML]{F3F2F0}}c }
        \multicolumn{2}{c}{\cellcolor[HTML]{359830}{\color[HTML]{F3F2F0} \textbf{$x\to 1^{-}$}}} \\
        {\color[HTML]{F3F2F0} \textbf{$x$}}      & \textbf{$y$} \\
        {\color[HTML]{FFFEFE} \textbf{$0$}}      & $2$          \\
        {\color[HTML]{FFFEFE} \textbf{$0,5$}}    & $2,5$        \\
        {\color[HTML]{FFFEFE} \textbf{$0,9$}}    & $2,9$        \\
        {\color[HTML]{FFFEFE} \textbf{$0,99$}}   & $2,99$       \\
        {\color[HTML]{FFFEFE} \textbf{$0,999$}}  & $2,999$      \\
        {\color[HTML]{FFFEFE} \textbf{$0,9999$}} & $2,9999$    
        \end{tabular}%
        \end{table}
        \vspace{-0.2cm}
        \begin{table}[]
          \centering
          \begin{tabular}{
          >{\columncolor[HTML]{359830}}c 
          >{\columncolor[HTML]{F3F2F0}}c }
          \multicolumn{2}{c}{\cellcolor[HTML]{359830}{\color[HTML]{F3F2F0} \textbf{$x\to 1^{+}$}}} \\
          {\color[HTML]{F3F2F0} \textbf{$x$}}      & \textbf{$y$} \\
          {\color[HTML]{FFFEFE} \textbf{$2$}}      & $4$          \\
          {\color[HTML]{FFFEFE} \textbf{$1,4$}}    & $3,4$        \\
          {\color[HTML]{FFFEFE} \textbf{$1,1$}}    & $3,1$        \\
          {\color[HTML]{FFFEFE} \textbf{$1,01$}}   & $3,01$       \\
          {\color[HTML]{FFFEFE} \textbf{$1,001$}}  & $3,001$      \\
          {\color[HTML]{FFFEFE} \textbf{$1,0001$}} & $3,0001$    
          \end{tabular}
          \end{table}
    \end{column}
  \end{columns}
\end{frame}

\begin{frame}
  \begin{columns}[onlytextwidth]
    \begin{column}{0.75\textwidth}\vspace{-0.5cm}
      \begin{example}
        Calcular o limite solicitado.
      \end{example}
      \begin{enumerate}\setcounter{enumi}{1}\small
        \item $\displaystyle\lim_{x\to -3}{h(x)}$, para $h(x) = \begin{cases}x+3, &\mbox{ se } x\not=-3 \\ 4, &\mbox{ se } x = -3\end{cases}$\vspace{0.3cm}
      \end{enumerate}
      \only<2>{
        \begin{columns}[onlytextwidth]
          \begin{column}{0.49\textwidth}\small\centering
            \textbf{Limite Lateral pela Esquerda}
            \begin{equation*}
              \lim_{x\to -3^{-}}{f(x)} = 0
            \end{equation*}
          \end{column}
          \begin{column}{0.49\textwidth}\small\centering
            \textbf{Limite Lateral pela Direita}
            \begin{equation*}
              \lim_{x\to -3^{+}}{f(x)} = 0
            \end{equation*}
          \end{column}
        \end{columns}
        \begin{itemize}\small
          \item Como os limites laterais são idênticos, temos que
          \begin{equation*}
            \lim_{x\to -3}{f(x)} = 0
          \end{equation*}
          \item Note que $f(-3) = 4$, valor \textbf{distinto} do limite
          \item Em geral, não é verdade que o limite seja igual ao valor da função\textbf{!}
        \end{itemize}
      }
    \end{column}
    \begin{column}<2>{0.25\textwidth}\vspace{-0.6cm}
      \begin{table}[]
        \centering
        \begin{tabular}{
        >{\columncolor[HTML]{359830}}c 
        >{\columncolor[HTML]{F3F2F0}}c }
        \multicolumn{2}{c}{\cellcolor[HTML]{359830}{\color[HTML]{F3F2F0} \textbf{$x\to -3^{-}$}}} \\
        {\color[HTML]{F3F2F0} \textbf{$x$}}      & \textbf{$y$} \\
        {\color[HTML]{FFFEFE} \textbf{$-3,5$}}   & $-0,5$       \\
        {\color[HTML]{FFFEFE} \textbf{$-3,3$}}   & $-0,3$       \\
        {\color[HTML]{FFFEFE} \textbf{$-3,1$}}   & $-0,1$       \\
        {\color[HTML]{FFFEFE} \textbf{$-3,01$}}  & $-0,01$      \\
        {\color[HTML]{FFFEFE} \textbf{$-3,001$}} & $-0,001$    
        \end{tabular}
        \end{table}
        \vspace{-0.2cm}
        \begin{table}[]
          \centering
          \begin{tabular}{
          >{\columncolor[HTML]{359830}}c 
          >{\columncolor[HTML]{F3F2F0}}c }
          \multicolumn{2}{c}{\cellcolor[HTML]{359830}{\color[HTML]{F3F2F0} \textbf{$x\to -3^{+}$}}} \\
          {\color[HTML]{F3F2F0} \textbf{$x$}}      & \textbf{$y$} \\
          {\color[HTML]{FFFEFE} \textbf{$-2,5$}}    & $0,5$          \\
          {\color[HTML]{FFFEFE} \textbf{$-2,8$}}    & $0,2$        \\
          {\color[HTML]{FFFEFE} \textbf{$-2,9$}}    & $0,1$        \\
          {\color[HTML]{FFFEFE} \textbf{$-2,99$}}   & $0,01$       \\
          {\color[HTML]{FFFEFE} \textbf{$-2,999$}}  & $0,001$                \end{tabular}
          \end{table}
    \end{column}
  \end{columns}
\end{frame}
