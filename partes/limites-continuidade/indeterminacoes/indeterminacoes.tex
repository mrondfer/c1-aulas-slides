\section{Indeterminações}

\begin{frame}
  \begin{definition}[Expressões Indeterminadas]
    Damos o nome de indeterminação àqueles limites que, \emph{a priori}, não sabemos o resultado.
  \end{definition}
  \begin{columns}[onlytextwidth]
    \begin{column}{0.49\textwidth}
      \begin{example-highlight}
        São exemplos de indeterminações expressões que resultam em casos de:
        \begin{equation*}
          \dfrac{0}{0}, \dfrac{\infty}{\infty}, \infty-\infty, 0\cdot \infty, 0^0, \infty^0, 1^{\infty}
        \end{equation*}
      \end{example-highlight}
    \end{column}
    \begin{column}{0.49\textwidth}
    \end{column}
  \end{columns}
\end{frame}

\begin{frame}
  \begin{example}
    Determine os limites:
  \end{example}
  \begin{enumerate}
    \item $\displaystyle\lim_{x\to -\infty}\left(\dfrac{1}{x^3-8}-\dfrac{1}{x-2}\right)$
    \item$\displaystyle\lim_{x\to +\infty}(\sqrt{x^2+3x+2}-x)$
  \end{enumerate}
\end{frame}

\begin{frame}
  \begin{example}
    Determine os limites a seguir:
  \end{example}
  \begin{enumerate}
    \item $\displaystyle\lim_{x\to -2}\dfrac{x^3-3x+2}{x^2-4}$
    \item $\displaystyle\lim_{t\to 2}\dfrac{t^5-32}{3t-6}$
  \end{enumerate}
\end{frame}

\begin{frame}
  \begin{example}
    Calcule os limite das funções irracionais abaixo.
  \end{example}
  \begin{enumerate}
    \item $\displaystyle\lim_{x\to 0}\dfrac{\sqrt{x+2}-\sqrt{2}}{x}$
    \item $\displaystyle\lim_{x\to 9}\dfrac{x^2-81}{\sqrt{x}-3}$
  \end{enumerate}
\end{frame}
