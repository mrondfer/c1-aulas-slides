\section{Limites Fundamentais}

\begin{frame}
  \begin{theorem}[Limite Fundamental Trigonométrico -- Seno]
    É válido o limite:
    \begin{equation*}
      \lim_{x\to 0} \dfrac{\sen{(x)}}{x} = 1
    \end{equation*}
  \end{theorem}
  \begin{columns}[onlytextwidth]
    \begin{column}{0.49\textwidth}\vspace{-0.7cm}
      \begin{figure}
        \includefigure[width=0.99\textwidth]{limite-fundamental-seno.pdf}
      \end{figure}
    \end{column}
    \begin{column}{0.49\textwidth}
      \begin{itemize}
        \item Para $x$ no primeiro quadrante, temos as áreas:
        \begin{equation*}
          \frac{\tan{x}\cdot 1}{2} \geq \frac{\pi\cdot 1^{2}\cdot x}{2\pi} \geq \frac{\sen{x}\cdot 1}{2}
        \end{equation*}
        
        \item Ao dividir por $\frac{2}{\sen{x}}$ e simplificar:
        \begin{equation*}
          \cos{x} \leq \frac{\sen{x}}{x} \leq 1
        \end{equation*}

        \item Basta aplicar o teorema do confronto quando $x\to 0$.
      \end{itemize}
    \end{column}
  \end{columns}
\end{frame}

\begin{frame}
  \begin{example}
    Calcule os limites trigonométricos a seguir.
  \end{example}
  \begin{enumerate}
    \item $\displaystyle\lim_{x\to 0}\dfrac{\sen{(2x)}}{x}$
    \item $\displaystyle\lim_{x\to 4}\dfrac{\sen{(8-2x)}}{4x-16}$
  \end{enumerate}
\end{frame}

\begin{frame}
  \begin{theorem}[Limite Fundamental Trigonométrico -- Cosseno]
    É válido o limite:
    \begin{equation*}
      \lim_{x\to 0} \dfrac{1-\cos{(x)}}{x} = 0
    \end{equation*}
  \end{theorem}
\end{frame}

\begin{frame}
  \begin{theorem}[Limite Fundamental Exponencial]
    É válido o limite:
    \begin{equation*}
      \lim_{x\to \pm \infty} {\left(1+\dfrac{1}{x}\right)}^x=e
    \end{equation*}
  \end{theorem}
  \begin{figure}\vspace*{-0.5cm}
    \includefigure[width=0.98\textwidth]{limite-fundamental-exponencial.pdf}
  \end{figure}
\end{frame}

\begin{frame}
  \begin{example}
    Calcule o limite:
    \begin{equation*}
      \displaystyle\lim_{x\to +\infty} {\left(1+\dfrac{3}{x}\right)}^{2x}
    \end{equation*}
  \end{example}
\end{frame}

\begin{frame}
  \begin{theorem}[\phantom{ç}]
    Para todo $k\neq 0$, são válidos os seguintes limites:
    \begin{enumerate}
      \item $\displaystyle\lim_{x\to 0}\dfrac{\sen{(kx)}}{kx}=1$
      \item $\displaystyle\lim_{x\to 0}\dfrac{1-\cos{(kx)}}{kx}=0$
      \item $\displaystyle\lim_{x\to \pm \infty} {\left(1+\dfrac{k}{x}\right)}^{x}=e^k$
      \item $\displaystyle\lim_{x\to 0} {(1+kx)}^{\frac{1}{x}}=e^{k}$
      \item $\displaystyle\lim_{x\to 0} \dfrac{a^{kx}-1}{kx}=\ln{a}$
    \end{enumerate}
  \end{theorem}
\end{frame}

\begin{frame}
  \begin{example}
    Calcule os limites a seguir.
  \end{example}
  \begin{enumerate}
    \item $\displaystyle\lim_{x\to 0}\dfrac{1-\cos{x}}{x^2}$
    \item $\displaystyle\lim_{x\to +\infty}{\left(\dfrac{x-2}{x-3}\right)}^x$
    \item $\displaystyle\lim_{x\to 0}\dfrac{e^{2x}-1}{x}$
    \item $\displaystyle\lim_{x\to 0}\dfrac{e^{x}-1}{\sen{x}}$
  \end{enumerate}
\end{frame}
