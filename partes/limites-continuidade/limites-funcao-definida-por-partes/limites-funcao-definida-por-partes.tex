\section{Limites de Funções Definidas por Partes}

\begin{frame}
  \begin{example}
    Considere a função
    \begin{equation*}
      f(x) = \begin{cases}
        \dfrac{1}{x+2}, &\mbox{se } x < -2\\
        x^{2} - 5, &\mbox{se } -2 < x \leq 3\\
        \sqrt{x + 13}, &\mbox{se } x>3
      \end{cases}
    \end{equation*}
    para avaliar os limites solicitados.
  \end{example}
  \begin{columns}[onlytextwidth]
    \begin{column}{0.3\textwidth}
      \begin{enumerate}
        \item $\displaystyle\lim_{x\to -2}f(x)$
        \item $\displaystyle\lim_{x\to 0}f(x)$
        \item $\displaystyle\lim_{x\to 3}f(x)$
      \end{enumerate}
    \end{column}
    \begin{column}{0.7\textwidth}
      \begin{highlight}
        \textbf{Observação:}
        \begin{itemize}
          \item Devemos ter cuidado ao utilizarmos as propriedades algébricas, visto que o comportamento da função pode mudar drasticamente nos pontos onde a lei é alterada;
          \item Nesse sentido, nos pontos onde trocamos de uma sentença para a outra, devemos sempre avaliar os limites laterais para determinar o limite.
        \end{itemize}
      \end{highlight}
    \end{column}
  \end{columns}
\end{frame}