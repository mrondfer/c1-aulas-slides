\section{Derivação Implícita}

\begin{frame}
  \begin{columns}[onlytextwidth]
    \begin{column}{0.49\textwidth}\vspace*{-0.5cm}\small
      \begin{itemize}
        \item Até agora temos trabalhado com funções \textbf{explicitamente} definidas;
        \item Ou seja, funções da forma
        \begin{equation*}
          y = f(x),
        \end{equation*}
        na qual estabelece explicitamente a variável dependente $y$ em termos da variável independente $x$. Em outras palavras, {\color{black}a variável $y$ aparece sozinha em um dos lados da equação;}
        \item No entanto, algumas funções são definidas através de equações que não deixam exposta a relação entre as variáveis $x$ e $y$. Em tais situações, {\color{black} a variável dependente $y$ \textbf{não} se encontra sozinha em um dos lados da equação.}
      \end{itemize}
    \end{column}
    \begin{column}{0.49\textwidth}\vspace*{-0.5cm}
      \begin{definition}[Função Implícita]
        Dizemos que uma dada equação em $x$ e $y$ define a função $f$ \textbf{implicitamente} se o gráfico de $y=f(x)$ coincidir com alguma porção do gráfico da equação.
      \end{definition}
      \begin{example-highlight}
        Temos que a função\vspace*{-0.2cm}
        \begin{equation*}
          y = \frac{x-1}{x+1}\vspace*{-0.2cm}
        \end{equation*}
        define $y$ explicitamente em termos de $x$.
      \end{example-highlight}
      \begin{example-highlight}
        Temos que a equação\vspace*{-0.2cm}
        \begin{equation*}
          yx + y + 1 = x\vspace*{-0.2cm}
        \end{equation*}
        define $y$ implicitamente em termos de $x$.
      \end{example-highlight}
    \end{column}
  \end{columns}
\end{frame}

\begin{frame}
  \begin{columns}[onlytextwidth]
    \begin{column}{0.49\textwidth}\vspace*{-0.5cm}
      \begin{itemize}
        \item Temos no exemplo que
        \begin{equation*}
          yx + y + 1 = x
        \end{equation*}
        define implicitamente $y$ em função de $x$.
        \item Uma pergunta \emph{natural} seria: qual é a derivada de $y(x)$?
        \item Dado que
        \begin{equation*}
          y = \frac{x-1}{x+1},
        \end{equation*}
        podemos usar a regra do quociente para obter a derivada desejada;
        \item De fato, temos que
        \begin{equation*}
          y^{\prime} = \frac{(x+1)\cdot 1 - (x-1)\cdot 1}{(x+1)^{2}} = \frac{2}{(x+1)^{2}}
        \end{equation*}
        \item Podemos, contudo, obter o mesmo resultado sem determinar $y$ explicitamente em função de $x$;
      \end{itemize}
    \end{column}
    \begin{column}{0.49\textwidth}\vspace*{-0.5cm}
      \begin{itemize}
        \item Para isso, vamos derivar os dois lados da equação que define $y$
        \begin{equation*}
          \frac{d}{dx}\left[yx + y + 1\right] = \frac{d}{dx}\left[x\right]
        \end{equation*}
        \item Pelas propriedades algébricas das derivadas, temos
        \begin{equation*}
          \frac{d}{dx}\left[yx\right] + \frac{d}{dx}\left[y\right] + \frac{d}{dx}\left[1\right] = \frac{d}{dx}\left[x\right]
        \end{equation*}
        \item \textbf{Supondo} que $y=f(x)$, temos que
        \begin{equation*}
          y\frac{d}{dx}\left[x\right] + x\frac{d}{dx}\left[y\right] + 0 = 1
        \end{equation*}
        \item De onde podemos obter
        \begin{equation*}
          y + xy^{\prime} = 1 \Rightarrow y^{\prime} = \frac{1-y}{x+1}
        \end{equation*}
        \item Temos que a derivada obtida é definida \textbf{implicitamente}!
      \end{itemize}
    \end{column}
  \end{columns}
\end{frame}

\begin{frame}
  \begin{columns}[onlytextwidth]
    \begin{column}{0.5\textwidth}\vspace*{-0.5cm}
      \begin{example}
        Determine a reta tangente à circunferência
        \begin{equation*}
          x^{2} + y^{2} = 25
        \end{equation*}
        no ponto $P(3,\,-4)$.
      \end{example}
    \end{column}
    \begin{column}{0.5\textwidth}
      \begin{figure}
        \includefigure[width=0.9\textwidth]{exemplo1.pdf}
      \end{figure}
    \end{column}
  \end{columns}
\end{frame}

\subsection{Derivada das Potências de x}
\begin{frame}
  \begin{theorem}[Derivada das Potências de $x$]
    Seja $f:I\rightarrow\mathbb{R}$ a função definida por $f(x) = x^{r}$, com $r\in\mathbb{Q}$. Então, 
    \begin{equation*}
      f^{\prime}(x) = r\cdot x^{r-1}.
    \end{equation*}
  \end{theorem}
  \begin{columns}[onlytextwidth]
    \begin{column}{0.55\textwidth}
      \begin{example-highlight}
        Determine as derivadas das funções a seguir.
        \begin{enumerate}
          \item $y = -\dfrac{1}{\sqrt[3]{x}}$
          \item $f(x) = \sqrt{25-x^{2}}$
        \end{enumerate}
      \end{example-highlight}
    \end{column}
    \begin{column}{0.43\textwidth}\vspace{-0.6cm}
    \end{column}
  \end{columns}
\end{frame}
