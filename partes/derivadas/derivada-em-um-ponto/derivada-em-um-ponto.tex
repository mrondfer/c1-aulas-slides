\section{Derivada em Um Ponto}
\subsection{Limites de Retas Secantes}
\begin{frame}
  \begin{columns}[onlytextwidth]
    \begin{column}{0.5\textwidth}\vspace*{-0.5cm}
      \begin{itemize}
        \item Considere $y=f(x)$, $x_{0}\in D(f)$ e $\Delta x\in\mathbb{R}$
        \item E a \textbf{reta secante} $r$ ao gráfico da $f$ em\vspace{-0.2cm}
        \begin{equation*}
          P_{0}(x_{0},\,f(x_{0})) \mbox{ e } P(x_{0} + \Delta x,\,f(x_{0} + \Delta x))
        \end{equation*}
        \item Com isso, temos
        \begin{itemize}
          \item[$\to$] $\Delta x$: \emph{incremento} na variável $x$
          \item[$\to$] $\Delta y = f(x_0+\Delta x) - f(x_{0})$
          \item[$\to$]  $\tan\beta = \dfrac{\Delta y}{\Delta x}$
        \end{itemize}
        \item Observe a \textbf{reta tangente} $t$ ao gráfico da $f$ no ponto $P_{0}$
        \item Ao passo que $\Delta x \to 0$, temos que $P \to P_{0}$, $r \to t$, $\beta \to \alpha$ e, por fim,
        \begin{equation*}
          \begin{split}
            \tan{\alpha} &= \lim_{\Delta x\to 0}\frac{\Delta y}{\Delta x} \\ &= \lim_{\Delta x\to 0}\frac{f(x_0+\Delta x) - f(x_{0})}{\Delta x}
          \end{split}
        \end{equation*}
      \end{itemize}
    \end{column}
    \begin{column}{0.5\textwidth}\vspace*{-0.5cm}
      \begin{figure}
        \includefigure[width=\textwidth]{secante-tangente.pdf}
      \end{figure}
    \end{column}
  \end{columns}
\end{frame}

\subsection{A Derivada em um Ponto}
\begin{frame}
  \begin{columns}[onlytextwidth]
    \begin{column}{0.5\textwidth}\vspace{-0.5cm}
      \begin{definition}[Derivada]
        Seja $f$ uma função definida em um intervalo aberto $I$ contendo $x_{0}$. A \textbf{derivada} da função $f$ no ponto $(x_{0},\,f(x_{0}))$ é o número real
        \begin{equation*}
          f^{\prime}(x_{0}) = \lim_{h\to 0}\frac{f(x_{0}+h)-f(x_{0})}{h}
        \end{equation*}
        \emph{desde que} o limite exista.
      \end{definition}
      \begin{highlight}
        \begin{itemize}
          \item Caso o limite exista, dizemos que $f$ é derivável (ou diferenciável) em $x_{0}$;
          \item Nesse caso, temos que\vspace*{-0.26cm}
          \begin{equation*}
            f^{\prime}(x_{0}) = \tan{\alpha},\vspace*{-0.26cm}
          \end{equation*}
          onde $\alpha$ é a \textbf{inclinação da reta tangente} ao gráfico da $f$ no ponto $(x_{0},\,f(x_{0}))$
        \end{itemize}
      \end{highlight}
    \end{column}
    \begin{column}{0.5\textwidth}\vspace{-0.5cm}
      \begin{itemize}
        \item Definições alternativas\small
        \begin{equation*}
          f^{\prime}(x_{0}) = \lim_{\Delta x\to 0}\frac{f(x_{0}+\Delta x)-f(x_{0})}{\Delta x}
        \end{equation*}
        \begin{equation*}
          f^{\prime}(x_{0}) = \lim_{x\to x_{0}}\frac{f(x)-f(x_{0})}{x - x_{0}}
        \end{equation*}
        \begin{equation*}
          f^{\prime}(x_{0}) = \lim_{x\to x_{0}}\frac{y - y_{0}}{x - x_{0}}
        \end{equation*}
        \begin{equation*}
          f^{\prime}(x_{0}) = \lim_{\Delta x\to 0}\frac{\Delta y}{\Delta x}
        \end{equation*}
      \end{itemize}
      \begin{itemize}
        \item Notações alternativas
        \begin{itemize}
          \item[$\to$] Lagrange: $f^{\prime}(x_{0})$, $y^{\prime}(x_{0})$
          \item[$\to$] Leibniz: $\dfrac{dy}{dx}(x_{0})$, $\dfrac{df}{dx}(x_{0})$
          \item[$\to$] Newton: $\dot{x}(t_{0}) = \dfrac{dx}{dt}(t_{0})$
        \end{itemize}
      \end{itemize}
    \end{column}
  \end{columns}
\end{frame}

\begin{frame}
  \begin{columns}[onlytextwidth]
    \begin{column}{0.5\textwidth}\vspace{-0.5cm}
      \begin{theorem}[Inclinação da Reta Tangente]
        Seja $f$ uma função definida em um intervalo aberto $I$ contendo $x_{0}$ para a qual a derivada
        \begin{equation*}
          f^{\prime}(x_{0}) = \lim_{h\to 0}\frac{f(x_{0}+h)-f(x_{0})}{h}
        \end{equation*}
        exista. Então, a inclinação $m$ da reta tangente ao gráfico da função $f$ no ponto de coordenadas $(x_{0},\,f(x_{0}))$ é dada por
        \begin{equation*}
          m = f^{\prime}(x_{0}).
        \end{equation*}
      \end{theorem}
      \begin{highlight}
        \textbf{E a reta tangente é dada por:}
        \begin{equation*}
          y = y_{0} + m\cdot(x-x_{0})\vspace*{-0.2cm}
        \end{equation*}
      \end{highlight}
    \end{column}
    \begin{column}{0.5\textwidth}\vspace{-0.5cm}
      \begin{figure}
        \includefigure[width=\textwidth]{secante-tangente.pdf}
      \end{figure}
    \end{column}
  \end{columns}
\end{frame}

\begin{frame}
  \begin{columns}[onlytextwidth]
    \begin{column}{0.49\textwidth}\vspace{-0.5cm}
      \begin{example}
        Determine o valor da derivada de
        \begin{equation*}
          f(x) = x^{2} - 4
        \end{equation*}
        nos pontos $x_{0} = 1$ e $x_{1}=-2$. Faça a representação das retas tangentes ao gráfico nos pontos dados.
      \end{example}
    \end{column}
    \begin{column}{0.49\textwidth}\vspace{-0.8cm}
      \begin{figure}
        \includefigure<1>[width=\textwidth]{exemplo1-vazio.pdf}
        \includefigure<2>[width=\textwidth]{exemplo1.pdf}
      \end{figure}
    \end{column}
  \end{columns}
\end{frame}

\begin{frame}
  \begin{columns}[onlytextwidth]
    \begin{column}{0.49\textwidth}\vspace{-0.5cm}
      \begin{example}
        Encontre uma equação para a reta \\tangente à curva de equação
        \begin{equation*}
          y = \frac{2}{x}
        \end{equation*}
        no ponto de coordenadas $(2,\,1)$. Esboce a curva e a reta tangente.
      \end{example}
    \end{column}
    \begin{column}{0.49\textwidth}
      \begin{figure}
        \includefigure<1>[width=\textwidth]{exemplo2-vazio.pdf}
        \includefigure<2>[width=\textwidth]{exemplo2.pdf}
      \end{figure}
    \end{column}
  \end{columns}
\end{frame}

\begin{frame}
  \begin{columns}[onlytextwidth]
    \begin{column}{0.49\textwidth}\vspace{-0.5cm}
      \begin{example}
        Encontre as inclinações das retas tangentes à curva
        \begin{equation*}
          y = \sqrt{x}
        \end{equation*}
        em $x_{0} = 1$ e $x_{1} = 2$. Esboce a curva e as retas tangentes.
      \end{example}
    \end{column}
    \begin{column}{0.49\textwidth}
      \begin{figure}
        \includefigure<1>[width=\textwidth]{exemplo3-vazio.pdf}
        \includefigure<2>[width=\textwidth]{exemplo3.pdf}
      \end{figure}
    \end{column}
  \end{columns}
\end{frame}
