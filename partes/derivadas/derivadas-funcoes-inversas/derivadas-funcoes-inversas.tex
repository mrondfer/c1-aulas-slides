\section{Derivadas das Funções Logarítmicas e Trigonométricas Inversas}

\subsection{Derivada das Funções Logarítmicas}
\begin{frame}
  \begin{theorem}[Derivada das Funções Logarítmicas]
    Dada a função logarítmica $f(x) = \log_{a}{x}$, temos que 
    \begin{equation*}
      f^{\prime}(x) = \frac{1}{x\cdot\ln{a}}
    \end{equation*}
    para todo $x>0$.
  \end{theorem}
  \begin{columns}[onlytextwidth]
    \begin{column}{0.55\textwidth}
      \begin{example-highlight}
        Determine as derivadas das funções a seguir.
        \begin{enumerate}
          \item $f(x) = \ln{(x^{2} + 1)}$
          \item $\displaystyle g(x) = \left(\frac{1}{2}\right)^{x}$
        \end{enumerate}
      \end{example-highlight}
    \end{column}
    \begin{column}{0.43\textwidth}
    \end{column}
  \end{columns}
\end{frame}

\subsection{Derivada das Funções Trigonométricas Inversas}
\begin{frame}
  \begin{theorem}[Derivada do Arco Seno]
    Dada a função arco seno $f(x) = \asen{x}$, temos que 
    \begin{equation*}
      f^{\prime}(x) = \frac{1}{\sqrt{1-x^{2}}}
    \end{equation*}
    para todo $x$ no intervalo $(-1,\,1)$.
  \end{theorem}
  \begin{columns}[onlytextwidth]
    \begin{column}{0.55\textwidth}
      \begin{example-highlight}
        Determine as derivadas das funções a seguir.
        \begin{enumerate}
          \item $f(x) = e^{\asen{x^{2}}}$
          \item $g(x) = \acos{x}$
          \item $h(x) = \atan{x}$
        \end{enumerate}
      \end{example-highlight}
    \end{column}
    \begin{column}{0.43\textwidth}
    \end{column}
  \end{columns}
\end{frame}

\begin{frame}
  \begin{theorem}[Derivada da Função Arco Cosseno]
    Dada a função arco cosseno $f(x) = \acos{x}$, temos que 
    \begin{equation*}
      f^{\prime}(x) = -\frac{1}{\sqrt{1-x^{2}}}
    \end{equation*}
    para todo $x$ no intervalo $(-1,\,1)$.
  \end{theorem}
  \vfill
  \begin{theorem}[Derivada da Função Arco Tangente]
    Dada a função arco tangente $f(x) = \atan{x}$, temos que 
    \begin{equation*}
      f^{\prime}(x) = \frac{1}{1+x^{2}}
    \end{equation*}
    para todo $x$ real.
  \end{theorem}
\end{frame}

