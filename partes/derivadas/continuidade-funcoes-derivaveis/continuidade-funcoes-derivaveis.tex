\section{Continuidade das Funções Deriváveis}

\subsection{Diferenciabilidade Implica Continuidade}
\begin{frame}
  \begin{theorem}[Toda Função Derivável é Contínua]
    Seja $f$ uma função derivável em $x_{0}$. Então, $f$ é contínua em $x_{0}$.
  \end{theorem}
  % \textbf{Demonstração:}
  % \begin{equation*}
  %   f(x) - f(x_{0}) = \frac{f(x) - f(x_{0})}{x-x_{0}}\cdot (x-x_{0}) \Rightarrow \lim_{x\rightarrow x_{0}}\left[f(x) - f(x_{0})\right] = f^{\prime}(x_{0})\cdot \lim_{x\rightarrow x_{0}} (x-x_{0}) = 0
  % \end{equation*}
  % Logo, temos a relação
  % \begin{equation*}
  %   \lim_{x\rightarrow x_{0}}\left[f(x) - f(x_{0})\right] = 0 \Leftrightarrow \lim_{x\rightarrow x_{0}}f(x) = f(x_{0})
  % \end{equation*}
  \vspace*{3cm}
  \visible<2>{
    \begin{highlight}
      \textbf{Algumas observações sobre este teorema}
    \begin{itemize}
      \item \textbf{Corolário:} Se $f$ não é contínua em $x_{0}$, então ela não é derivável em $x_{0}$;
      \item Veremos que uma função \textbf{nunca} é diferenciável:
      \begin{itemize}
        \item[$\rightarrow$] Nos pontos de descontinuidade;
        \item[$\rightarrow$] Nos pontos em que o seu gráfico possui uma quina (curva não suave);
        \item[$\rightarrow$] Nos pontos em que a reta tangente é vertical.
      \end{itemize}
    \end{itemize}
    \end{highlight}
  }
\end{frame}

\begin{frame}
  \begin{columns}[onlytextwidth]
    \begin{column}{0.49\textwidth}\vspace{-0.5cm}
      \begin{example}
        A função definida pela lei
        \begin{equation*}
          h(x) = \begin{cases}
            x^{2} - 4,\mbox{ se } x\leq 1\\ 2x - 4,\mbox{ se } x > 1
          \end{cases}
        \end{equation*}
        não é derivável em $x=1$.
      \end{example}
    \end{column}
    \begin{column}{0.5\textwidth}\vspace{-0.5cm}
      \begin{figure}
        \includefigure[width=\textwidth]{exemplo1.pdf}
      \end{figure}
    \end{column}
  \end{columns}
\end{frame}

\begin{frame}
  \begin{columns}[onlytextwidth]
    \begin{column}{0.49\textwidth}\vspace{-0.5cm}
      \begin{example}
        A função $f(x) = |x|$ é contínua em $x = 0$, mas não é derivável nesse ponto.
      \end{example}
    \end{column}
    \begin{column}{0.5\textwidth}\vspace*{-0.5cm}
      \begin{figure}
        \includefigure[width=\textwidth]{exemplo2.pdf}
      \end{figure}
    \end{column}
  \end{columns}
\end{frame}

\begin{frame}
  \begin{columns}[onlytextwidth]
    \begin{column}{0.49\textwidth}\vspace{-0.5cm}
      \begin{example}
        A função $g(x) = \sqrt[3]{x}$ é contínua em $x = 0$, mas não é derivável nesse ponto.
      \end{example}
    \end{column}
    \begin{column}{0.5\textwidth}\vspace{-0.5cm}
      \begin{figure}
        \includefigure[width=\textwidth]{exemplo3.pdf}
      \end{figure}
    \end{column}
  \end{columns}
\end{frame}
