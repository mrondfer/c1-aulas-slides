\section{Regra de L'Hôpital}

\begin{frame}
  \begin{theorem}[Regra de L'Hôpital]
    Se tivermos que $\displaystyle\lim_{x\to x_{0}}\frac{f(x)}{g(x)}$ for uma forma indeterminada do tipo $\dfrac{0}{0}$ ou $\dfrac{\infty}{\infty}$, então
    \begin{equation*}
      \lim_{x\to x_{0}}\frac{f(x)}{g(x)} = \lim_{x\to x_{0}}\frac{f^{\prime}(x)}{g^{\prime}(x)},
    \end{equation*}
    caso o limite exista (sendo finito ou infinito).
  \end{theorem}
  \begin{columns}[onlytextwidth]
    \begin{column}{0.5\textwidth}
      \begin{highlight}
        \textbf{Observação:}
        \begin{enumerate}
          \item O mesmo também é válido quando $x\to x_{0}^{+}$, $x\to x_{0}^{-}$, $x\to +\infty$ ou $x\to -\infty$;
          \item A existência de $\displaystyle \lim_{x\to x_{0}}\frac{f(x)}{g(x)}$ \textbf{não garante} a existência de $\displaystyle \lim_{x\to x_{0}}\frac{f^{\prime}(x)}{g^{\prime}(x)}$\vspace*{-0.2cm}
        \end{enumerate}
      \end{highlight}
    \end{column}
    \begin{column}{0.5\textwidth}
    \end{column}
  \end{columns}
\end{frame}

\begin{frame}
  \begin{example}
    Calcule os limites a seguir.
  \end{example}
  \begin{columns}[onlytextwidth]
    \begin{column}{0.5\textwidth}
      \begin{enumerate}
        \item $\displaystyle\lim_{x\to 0}\frac{\sen{x}}{x}$
        \item $\displaystyle\lim_{x\to -2}\frac{x^{3}-3x+2}{x^{2}-4}$
        \item $\displaystyle\lim_{x\to 9}\frac{x^{2}-81}{\sqrt{x}-3}$
        \item $\displaystyle\lim_{x\to +\infty}\frac{\ln{2x}}{x^{2}}$
      \end{enumerate}
    \end{column}
    \begin{column}{0.5\textwidth}
      %
    \end{column}
  \end{columns}
\end{frame}

\begin{frame}
  \begin{example}
    Para $f(x) = x^{2}\cdot\sen{\dfrac{1}{x}}$ e $g(x) = x$, avalie o limite
    \begin{equation*}
      \lim_{x\to 0}\frac{f(x)}{g(x)}.
    \end{equation*}
  \end{example}
\end{frame}