\section{Derivada de Uma Função}
\begin{frame}
  \begin{columns}[onlytextwidth]
    \begin{column}{0.5\textwidth}\vspace{-0.5cm}
      \begin{definition}[Derivada de uma Função]
        Seja $f$ uma função definida em um intervalo aberto $I$. Definimos a derivada de $f$ como a função $f^{\prime}$ que associa a cada $x$ de $I$ o número
        \begin{equation*}
          f^{\prime}(x) = \lim_{h\rightarrow 0}\frac{f(x+h)-f(x)}{h}
        \end{equation*}
        desde que o limite exista.
      \end{definition}
      \begin{highlight}
        \textbf{Observações:}
        \begin{itemize}
          \item Diz-se que a função $f^{\prime}$ é a derivada da função $f$;
          \item O domínio da $f^{\prime}$ é o conjunto de todos os $x\in I$ para os quais o limite existe;
          \item Se $D(f^{\prime})=I$, então diz-se que $f$ é derivável em $I$.
        \end{itemize}
      \end{highlight}
    \end{column}
    \begin{column}{0.49\textwidth}\vspace{-0.5cm}
      \begin{highlight}
        \begin{itemize}
          \item Além disso, $f^{\prime}(x)$ representa o coeficiente angular da reta tangente ao gráfico da $f$ em qualquer ponto $(x,\,f(x))$ de seu domínio
          \begin{itemize}
            \item[$\rightarrow$] Temos, com isso, uma reta tangente para cada ponto do domínio.
          \end{itemize}
          \item A vantagem da função derivada $f^{\prime}(x)$ frente a derivada no ponto $f^{\prime}(x_{0})$ é a habilidade obtenção da derivada da $f$ em qualquer ponto em que seja possível calculá-la
        \end{itemize}
      \end{highlight}
    \end{column}
  \end{columns}
\end{frame}

\begin{frame}
  \begin{columns}[onlytextwidth]
    \begin{column}{0.48\textwidth}\vspace{-0.5cm}
      \begin{example}
        Usando a definição, determine a derivada das funções ao lado.
      \end{example}
    \end{column}
    \begin{column}{0.5\textwidth}\vspace{-0.5cm}
      \begin{enumerate}
        \item $g(x) = x^{2} - 4x + 3$
        \item $h(x) = 2^{x} - 3x$
      \end{enumerate}
    \end{column}
  \end{columns}
\end{frame}
