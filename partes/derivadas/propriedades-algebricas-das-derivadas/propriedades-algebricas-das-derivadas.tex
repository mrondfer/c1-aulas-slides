\section{Propriedades Algébricas das Derivadas}
\begin{frame}
  \begin{theorem}[Derivadas da Soma, da Diferença e do Produto por Escalar]
    Sejam $u$ e $v$ funções deriváveis em $x_{0}$ e $a$ um número real. Então, são válidas as propriedades a seguir:
    \begin{enumerate}
      \item Se $f(x) = a\cdot u(x)$, então a derivada de $f$ em $x_{0}$ é
      \begin{equation*}
        f^{\prime}(x_{0}) = a\cdot u^{\prime}(x_{0})
      \end{equation*}
      \item Se $f(x) = u(x) + v(x)$, então a derivada de $f$ em $x_{0}$ é
      \begin{equation*}
        f^{\prime}(x_{0}) = u^{\prime}(x_{0}) + v^{\prime}(x_{0})
      \end{equation*}
      \item Se $f(x) = u(x) - v(x)$, então a derivada de $f$ em $x_{0}$ é
      \begin{equation*}
        f^{\prime}(x_{0}) = u^{\prime}(x_{0}) - v^{\prime}(x_{0})
      \end{equation*}
    \end{enumerate}
  \end{theorem}
\end{frame}

\begin{frame}
  \begin{columns}[onlytextwidth]
    \begin{column}{0.55\textwidth}\vspace{-0.5cm}
      \begin{example}
        Determine as derivadas das funções a seguir.
        \begin{enumerate}
          \item $f(x) = 3x^{4} - 2x^{3} + 7x - 5$
          \item $g(x) = 3\cdot 4^{x} - 5x^2$
          \item $h(x) = x(2x-1)^{2}$
          % \item $i(x) = 6\sqrt{x} + 2\sqrt[3]{x} - 5\sqrt[4]{x^3}$
          % \item $j(x) = \frac{3}{x^5}-2\cdot 3^x + \frac{2}{\sqrt{x}}$
        \end{enumerate}
      \end{example}
    \end{column}
    \begin{column}{0.43\textwidth}\vspace{-0.5cm}
    \end{column}
  \end{columns}
\end{frame}

\begin{frame}
  \begin{example}[Derivada do Produto]
    Vamos considerar as funções $f(x) = -2x$ e $g(x) = x^4$. Qual é a derivada da função produto $h(x) = f(x)\cdot g(x)$?
  \end{example}
\end{frame}

\subsection{Derivada do Produto}
\begin{frame}
  \begin{theorem}[Derivada do Produto de Funções]
    Sejam $u$ e $v$ funções deriváveis em um dado ponto $x_{0}$. Então, a função produto $u\cdot v$ também é derivável em $x_{0}$ e a sua derivada é dada pela expressão
    \begin{equation*}
      \frac{d}{dx}\left[u(x_{0}) v(x_{0})\right]=u(x_{0}) \frac{d}{dx}\left[v(x_{0})\right]+v(x_{0}) \frac{d}{dx}\left[u(x_{0})\right]
    \end{equation*}
  \end{theorem}
  \begin{columns}[onlytextwidth]
    \begin{column}{0.5\textwidth}
      \begin{example-highlight}
        Calcule a derivada da funções a seguir.
        \begin{enumerate}
          \item $f(x)=(x+1)(2x^3-4)$
          \item $g(x) = -2x\cdot 3^{x}$
          \item $h(x) = e^{x}(x-1)(x+1)$
        \end{enumerate}
      \end{example-highlight}
    \end{column}
    \begin{column}{0.5\textwidth}
    \end{column}
  \end{columns}
  \vspace*{-0.2cm}
  \visible<2>{%
    \begin{highlight}
      \textbf{Observação:}
      \begin{itemize}
        \item A regra da cadeia pode ser ampliada para o produto de três, quatro, ..., $n$ funções.
      \end{itemize}
    \end{highlight}
  }
\end{frame}

\subsection{Derivada do Quociente}
\begin{frame}
  \begin{theorem}[Derivada do Quociente de Funções]
    Sejam $u$ e $v$ funções deriváveis em $x_{0}$. Então, a função quociente $\dfrac{u}{v}$ é derivável em $x_{0}$ e a sua derivada em $x_{0}$ é dada por pela expressão
    \begin{equation*}
      \frac{d}{dx}\left[\frac{u(x_{0})}{v(x_{0})}\right]=\frac{v(x_{0})\dfrac{d}{dx}\left[u(x_{0})\right] - u(x_{0})\dfrac{d}{dx}\left[v(x_{0})\right]}{\left[v(x_{0})\right]^{2}},
    \end{equation*}
    desde que $v(x_{0}) \not= 0$.
  \end{theorem}
  \begin{columns}[onlytextwidth]
    \begin{column}{0.5\textwidth}
      \begin{example-highlight}
        Calcule a derivada da função a seguir.
        \begin{enumerate}
          \item $f(x)=\dfrac{4x+2}{x^2+1}$.
          \item $f(x)=\dfrac{x\cdot 2^{x}}{e^{x}}$.
        \end{enumerate}
      \end{example-highlight}
    \end{column}
    \begin{column}{0.5\textwidth}
    \end{column}
  \end{columns}
\end{frame}

\subsection{Derivada das Potências de x}
\begin{frame}
  \begin{theorem}[Derivada das Potências de $x$]
    Seja $f:I\rightarrow\mathbb{R}$ a função definida por $f(x) = x^{n}$, com $n\in\mathbb{Z}$. Então, 
    \begin{equation*}
      f^{\prime}(x) = n\cdot x^{n-1}.
    \end{equation*}
  \end{theorem}
  \begin{columns}[onlytextwidth]
    \begin{column}{0.55\textwidth}
      \begin{example-highlight}
        Determine as derivadas das funções a seguir.
        \begin{enumerate}
          \item $f(x) = \dfrac{1}{x^{2}}$
          \item $y = -\dfrac{4}{x^{5}}$
        \end{enumerate}
      \end{example-highlight}
    \end{column}
    \begin{column}{0.43\textwidth}\vspace{-0.6cm}
    \end{column}
  \end{columns}
\end{frame}

\begin{frame}{}
    \begin{table}[H]
    \renewcommand{\arraystretch}{1.5}
    \footnotesize
    \begin{tabular}{ll}
    \multicolumn{1}{c}{\textbf{Função}}            & \multicolumn{1}{c}{\textbf{Derivada}}                                         \\ \hline
    \multicolumn{1}{|l|}{$f(x)=k$}                 & \multicolumn{1}{l|}{$f'(x)=0$}                                                \\ \hline
    \multicolumn{1}{|l|}{$f(x)=x^n$}               & \multicolumn{1}{l|}{$f'(x)=n \cdot x^{n-1}$}                                  \\ \hline
    \multicolumn{1}{|l|}{$f(x)=a^x$}               & \multicolumn{1}{l|}{$f'(x)=\ln a \cdot a^x$}                                  \\ \hline
    \multicolumn{1}{|l|}{$f(x)=a\cdot u(x)$}       & \multicolumn{1}{l|}{$f'(x)=a\cdot u'(x)$}                                     \\ \hline
    \multicolumn{1}{|l|}{$f(x)=u(x)\pm v(x)$}      & \multicolumn{1}{l|}{$f'(x)=u'(x)\pm v'(x)$}                                   \\ \hline
    \multicolumn{1}{|l|}{$f(x)=u(x)\cdot v(x)$}     & \multicolumn{1}{l|}{$f'(x)=u'(x)\cdot v(x)+u(x)\cdot v'(x)$}                  \\ \hline
    \multicolumn{1}{|l|}{$f(x)=\dfrac{u(x)}{v(x)}$} & \multicolumn{1}{l|}{$f'(x)=\dfrac{u'(x)\cdot v(x)-u(x)\cdot v'(x)}{[v(x)]^2}$} \\ \hline
    \end{tabular}
\end{table}
\end{frame}
